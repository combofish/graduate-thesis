%%==================================================
%% thanks.tex for DUT Thesis
%% version: 0.1
%% last update: Apr 27th, 2022
%%==================================================

\begin{thanks}
%学位论文中不得书写与论文工作无关的人和事(可以写家人),对导师的致谢要实事求是。\par
%对指导或协助指导完成论文的导师、资助基金或合同单位、提供帮助和支持的同志应在论文中做明确的说明并表示谢意。\par
%这部分内容不可省略。\par


落笔至此,感慨万千。三年如白驹过隙,眼下虽是论文的终章,却是未来的起点。
回首这几年,我曾憧憬过、迷茫过、失落过、拼搏过,经历了诸多起伏,收获了成长与收获。
在此,我要衷心感谢所有曾经帮助过我的人,也感谢自己的努力和坚持。


首先,特别感谢我的导师朴永日副教授。
朴老师的一丝不苟、开拓创新的科研思维和积极乐观态度深深激励与教导了我。
感谢朴老师引领我踏入计算机视觉领域,
朴老师在科研上的细致指导,帮助我规划研究方向,培养我的科研能力。
也感谢朴老师对我生活的关怀,在我失落时,朴老师给予了很多勉励和支持,让我重新振作面对困难和挫折。
同时,也深表感谢同组的张淼副教授,感谢她提供的写作思路和修改建议,
张老师的学术严谨与精益求精的工作作风让我受益匪浅。
两位导师是我读研期间最大的恩人,两位导师的教诲我将永铭心怀,能成为他们的学生,是我毕生的幸运。


其次,感谢所有关心帮助过我的同学们,能结识课题组中的同学是我一生中最大的财富。
感谢姜永耀师兄、王健师兄和吴为师兄在我初踏入实验室时的耐心指导;
感谢赵永帅师兄,陆晨阳师兄和刘廷位师兄在科研和工作上对我的支持;
感谢孙小飞师兄,许爽师姐对我实习工作上的帮助,缓解了我初入职场的忐忑,
让我更快的适应了工作环境;
感谢刘垒烨和尹纪浩对我竞赛上的帮助,不仅让我学到了很多新知识,也让我感受到团队的力量!
同时也祝愿姚臻彦、王治、
王书{\CJKfontspec{NotoSerifTC-Regular.otf} 垚}、
钟嘉龙、马宁等师弟师妹们在科研之路上一帆风顺。
感谢同门吴岚虎和李智玮,至今都在怀念咱们一起探讨科研的时光,
感谢你们的帮助,我们来日再相聚。


最后,由衷感谢我的家人对我所给予的支持,没有他们,我不可能走到今天。我也要感谢当初选择考研时的自己,给了未来自己另一种可能性;感谢在考研过程中我能够坚持孤独、夜以继日地学习;更要感谢在研究生阶段我能够保持坚持并且有所收获。未来将有更广阔的世界等待着我,希望我能记住最初的目标,不忘初心,继续前行。


从家到学校,短短二十公里之路,走过整整二十年。如今即将走向江湖,将来漫漫云霄,我们各自有着不同的归程!祝愿国家稳步复兴、富强兴盛,也希望自己一帆风顺、幸福安康!再次感谢所有关心帮助过我的人,我将继续努力,成为更优秀的自己。


%{\CJKfontspec{NotoSerifTC-Regular.otf} 想要修改字体的部分
%
%鑫
%犇
%焱
%懿
%燚
%垚
%}

\end{thanks}

