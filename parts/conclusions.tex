%\BiChapter{结论与展望}{Conclusions and Prospection}
%该部分主要包括三个部分:“结论”、“创新点”和“展望”。
%\BiSection{结论}{ Conclusions}
%结论是理论分析和实验结果的逻辑发展,是整篇论文的归宿。结论是在理论分析、试验结果的基础上,经过分析、推理、判断、归纳的过程而形成的总观点。结论必须完整、准确、鲜明、并突出与前人不同的新见解。
%\BiSection{创新点}{Highlights}
%创新点应该以分条列举的形式进行提出。\par
%(1) 以预报……模型,建立了….。 \par
%(2) 应用……方法,对颗粒受力情况进行了分析。\par
%(3) ……\par
%(4) ……
%\BiSection{展望}{Prospection}
%展望是对该研究课题存在的不足和有待改进的说明,是对未来研究的一种期待。


\BiChapter{结论与展望}{Conclusions and Prospection}


该部分主要包括三个部分:“结论”、“创新点”和“展望”。


\BiSection{结论}{ Conclusions}


结论是理论分析和实验结果的逻辑发展,是整篇论文的归宿。结论是在理论分析、试验结果的基础上,经过分析、推理、判断、归纳的过程而形成的总观点。结论必须完整、准确、鲜明、并突出与前人不同的新见解。


显著性目标检测旨在辨识图像中最引人注目的对象或区域,在计算机视觉领域占有关键地位,对图像理解具有重要作用。光场数据相比传统的二维数据包含更丰富的场景信息,适用于挑战性场景下显著性检测对高维信息的要求。然而,在实际场景中,光场的获取成本高、多线索信息处理复杂,以及耗时费力的像素级显著性标注,导致当前光场显著性目标检测数据稀缺,难以支撑深度模型训练所需的数据。针对上述挑战,本文通过高效利用光场信息和增加光场数据量两方面入手,探索了利用有限数据进行光场显著性目标检测的方法。主要研究内容如下:

%
%
(1)


(2)



\BiSection{创新点}{Highlights}
%
%
%创新点应该以分条列举的形式进行提出。\par
%(1) 以预报……模型,建立了….。 \par
%(2) 应用……方法,对颗粒受力情况进行了分析。\par
%(3) ……\par
%(4) ……
%
%


(1)
通过构建切片级嵌入式令牌表示,通过令牌之间的交叉注意力和移位操作,
建立了网络对于光场三维场景的感知。


(2)
通过建立全聚焦特征与焦点堆栈特征的聚焦聚焦匹配,
增强了焦点堆栈中显著物体表示,并抑制非显著的背景噪声影响。


(3)
提出视角增强注意力来促进跨焦点堆栈和全聚焦特征的融合,
在进行两个模态的交叉注意计算时,使用跨模态的显著性前景表达来增强权重矩阵,
降低了由于焦点堆栈中背景聚焦引起的注意力转移。


(4)
引入像素对比学习策略,使得网络在分辨像素类别的同时,也能够考虑
显著性像素区域的内在联系和与非显著性区域的整体差异。
增加了网络对显著性区域的辨识能力。



\BiSection{展望}{Prospection}
%
%展望是对该研究课题存在的不足和有待改进的说明,是对未来研究的一种期待。
%
%

光场信息独有的高维数据表示,使得光场显著性检测网络在一些复杂场景上相比RGB或RGBD的
显著性检测具有更大的优势,但也带来了更高的计算负担。
更加高效的光场信息提取方式和更为轻量的光场显著性检测网络具有更高的现实意义。
我们会在接下来的工作探索使用轻量级网络实现光场显著性目标检测。



















