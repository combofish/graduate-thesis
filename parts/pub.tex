%%==================================================
%% pub.tex for DUT Thesis
%% version: 0.1
%% last update: Apr 27th, 2022
%%==================================================

\begin{publications}
%\subsubsection*{\textbf{~~已发表论文}}
%\vspace{-10pt}
%\begin{enumerate}[label={[\arabic*]}]
%	\item\textbf{Zhao X}, Yin Z, Zhang B*, Yang Z. Experimental investigation of surface temperature non-uniformity in spray cooling [J]. \textbf{\textsl{International Journal of Heat and Mass Transfer}}, 2020, 146: 118819. (SCI: 000500371700033, EI: 20194207543161, IF: 4.346, 本学位论文第三章) 
%	\item\textbf{作者1}, 作者2, 作者3, 作者4*. 基于导热逆问题的间歇性喷雾研究, \textit{中国工程热物理学会多相流学术会议}, 2018, 北京. (本学位论文第四章)
%\end{enumerate}


\subsubsection*{\textbf{~~待发表论文}}
\vspace{-10pt}
\begin{enumerate}[label={[\arabic*]}]
%	\item\textbf{Zhang Q}, Chen S*, Yu H, Quan X*. Solar-driven simultaneously extracting clean water and pure NH3•H2O solution by carbonized wood. In Preparation/Under Review (本学位论文第二章)
	
	
	\item
%	\textbf{Liming Zhao}, Miao Zhang, Yongri Piao and Huchuan Lu.
	\textbf{Zhao Liming}, Zhang Miao, Piao Yongri* and Lu Huchuan.
	Focal Perception Transformer for Light Field Salient Object Detection.
	In Preparation/Under Review
%	IEEE Transactions on Image Processing, 2024
	(本学位论文第三章)
\end{enumerate}


\subsubsection*{\textbf{~~软件著作权}}
\vspace{-10pt}
\begin{enumerate}[label={[\arabic*]}]
%	\item\textbf{发明人1}, 发明人2, 发明人3. 多功能一次性压舌板: 中国. 发明类别: 发明专利. 授权公告号: 92214985.2 [P], 公开(或授权)日期: 1993.04.14.
%	\item\textbf{发明人1}, 发明人2. 发明人3. 气体恒温控制装置及混合气体节流系统: 中国. 发明类别: 发明专利. 授权公告号: CN 107562086 B, 授权公告日: 2020.02.14.
	
	\item\textbf{赵立明}, 朴永日, 卢湖川.
	显著性检测性能评估系统
	: 中国. 发明类别: 软件著作权. 
	授权公告号: 2023SR0592002, 
	授权公告日: 2023.06.07.
\end{enumerate}	


%\subsubsection*{\textbf{~~获得奖励}}
%\vspace{-10pt}
%\begin{enumerate}[label={[\arabic*]}]
%	\item “大型C/E复合材料构件高质高效加工关键技术及其工艺装备”, 机械工业科学技术奖-科技进步一等奖, 2013.10, 完成人排序: x/y(如1/3).
%\end{enumerate}
%\subsubsection*{\textbf{~~参与科研项目}}
%\vspace{-10pt}
%\begin{enumerate}[label={[\arabic*]}]
%	\item 国家自然科学基金面上项目(21476037): 微细通道内液滴运动行为的调控及混合与吸收过程强化机理的研究, 2015.01 – 2018.12, 负责人: 张三.
%\end{enumerate}	


\end{publications}
