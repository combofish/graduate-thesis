
%%==================================================
%% Ma Thesis.tex for DUT Thesis
%% version: 1.2
%% last update: Apr 27th, 2022
%%==================================================


\documentclass[twoside,doctor,hide]{DUT-thesis-grd}

% 需要删除
% 自定义宏
\def\todo{\textcolor{red}{TODO}}
% END 

% 自己额外导入的包
\usepackage{ulem}
% If you comment hyperref and then uncomment it, you should delete
% egpaper.aux before re-running latex.  (Or just hit 'q' on the first latex
% run, let it finish, and you should be clear).
%\usepackage[pagebackref=true,breaklinks=true,letterpaper=true,colorlinks,bookmarks=false]{hyperref}
%
%\usepackage[colorlinks,
%linkcolor=red,
%anchorcolor=blue,
%citecolor=green
%]{hyperref}

%\usepackage[dvipdfm,  %pdflatex,pdftex这里决定运行文件的方式不同
%pdfstartview=FitH,
%CJKbookmarks=true,
%bookmarksnumbered=true,
%bookmarksopen=true,
%colorlinks, %注释掉此项则交叉引用为彩色边框(将colorlinks和pdfborder同时注释掉)
%pdfborder=001,   %注释掉此项则交叉引用为彩色边框
%linkcolor=green,
%anchorcolor=green,
%citecolor=green
%]{hyperref}  

\usepackage{hyperref} 
\hypersetup{
	colorlinks=true,
	citecolor=blue,
	linkcolor=red,
	urlcolor=green}


\usepackage[UTF8]{ctex} % 使用 ctex 宏包处理中文
\usepackage{xeCJK} % 使用 xeCJK 宏包处理中文
%\setCJKmainfont{標楷體} % 设置繁体中文字体为標楷體

%\usepackage[colorlinks]{hyperref}
\usepackage{pifont}
\usepackage{color}
\usepackage{booktabs}
\usepackage{multirow}
\usepackage{color}
%\usepackage{subcaption}
%\usepackage{cite}
%\usepackage[numbers]{natbib}
\usepackage{soul}


%==============更改数学字体设置,Latin Modern Math 默认的的确有点细,看个人需要,下面提供一种方法,需要的可以取消注释=========%

% \usepackage[bold-style=ISO]{unicode-math} %采用unicode-math,可以直接输入Unicode公式,当然传统的输入就行
% \setmathfont{XITS Math}  %目前unicode-math 支持几种数学字体,具体用法可以查看帮助文档,这里采用类似times字体科学数学字体,可以取消注释对比


\begin{document}

%%%%%%%%%%%%%%%%%%%%%%%%%%%%%%
%% [-] 1,封面
%% [-] 2,原创声明
%% [-] 3,版权使用授权书
%% [ ] 4,摘要
%% [ ] 5,英文摘要
%% [-] 6,目录
%% [-] 7,英文目录

%% [ ] 8,图表索引

%% [ ] 9,正文
%% [ ] 9,第一章
%% [ ] 9,第二章
%% [ ] 9,第三章
%% [ ] 9,第四章
%% [ ] 10,结论


%% [-] 11,参考文献
%% [-] 12,发表文章目录
%% [-] 13,致谢
%%%%%%%%%%%%%%%%%%%%%%%%%%%%%%

% 封面绘制
%\maketitle

% 论文原创性声明和使用授权
%\makeDeclareOriginal
%\addcontentsline{toc}{chapter}{大连理工大学学位论文版权使用授权书}

%%%%%%%%%%%%%%%%%%%%%%%%%%%%%%
%% 前置部分
%%%%%%%%%%%%%%%%%%%%%%%%%%%%%%
\frontmatter

% 摘要
%%%==================================================
%% abstract.tex for DUT  Thesis
%% version: 0.1
%% last update: Apr 27th, 2022
%%==================================================

\begin{abstract}
%大连理工大学硕士研究生撰写学位论文应当符合写作规范和排版格式的要求,以下格式为研究生院依据国家标准和行业规范所编制的硕士学位论文模板,供硕士研究生参照使用。\par 论文摘要是学位论文的缩影,文字要简练、明确。内容要包括目的、方法、结果和结论。单位制一律换算成国际标准计量单位制,除特殊情况外,数字一律用阿拉伯数码。文中不允许出现插图,重要的表格可以写入。\par  摘要的主要内容为,简述全文的目的和意义、采用方法、主要研究内容和结论。\par 篇幅以一页为限,摘要正文后列出3-5个关键词,关键词与摘要之间空一行。\par“关键词:”是关键词部分的引导,不可省略。\par 关键词请尽量用《汉语主题词表》等词表提供的规范词。关键词之间用分号间隔,末尾不加标点。
%
%
%
%
	
	
显著性目标检测的目标在于识别图像中最引人注目的对象或区域,这是计算机视觉领域中一个重要的任务。现有的显著性目标检测算法根据输入数据的类型可以被分为三类:RGB、RGB-D和光场方法。与RGB和RGB-D数据相比,光场数据包含了更丰富的场景信息,可以满足对于复杂场景信息的需求。近年来,随着深度卷积神经网络的发展,它取代了传统的基于手工特征的算法,显著提升了光场显著性目标检测的性能。
%
%
%然而,实际应用中存在较高的光场数据获取成本、复杂的光场多线索信息处理以及耗时耗力的显著性像素级标注,这导致当前光场显著性目标检测数据稀缺,为深度模型提供足够支持的数据不足。为解决这些问题,本文从高效利用光场信息和增广光场数据两个角度出发,探索利用有限数据驱动的光场显著性目标检测方法。
%
%
然而,实际应用中存在复杂的光场信息提取以及跨模态的光场信息融合难等问题,
这导致了当前光场显著性检测深度模型难以有效辨别光场场景的的显著性物体表示。
%
%
为了解决这些问题,本文从焦点感知和视角增强两个角度出发,
探索基于聚焦感知的光场显著性检测方法。
本文的主要工作及创新点如下所述:


%%%%%%%%%%%%%%%%%%%%%%%%%%%%%%%%%%%%%%%%%%%%%
%
% 第一个工作点
%
%%%%%%%%%%%%%%%%%%%%%%%%%%%%%%%%%%%%%%%%%%%%%
(1)
%
%
%最近,光场显著性物体检测(LFSOD)因在复杂场景中利用丰富的光场线索取得显著改进而引起了越来越多的关注。虽然许多工作在这一领域取得了显著进展,但对其焦点特性的更深入洞察应该被开发。在这项工作中,我们提出了焦点感知变换器(FPT),可以高效地编码焦点堆栈内和全部焦点图像中的上下文。具体而言,我们引入了与焦点相关的令牌来总结图像特定特征,并且提出了令牌通信模块(TCM)来传达信息并促进空间上下文建模。通过精确编码的与焦点相关的令牌之间的信息交换,可以丰富每幅图像的特征并与其他图像相关联。我们还提出了焦点感知增强(FPE)策略,以帮助抑制嘈杂的背景信息。对四个广泛使用的基准数据集进行的大量实验证明,所提出的模型优于当前最先进的方法。我们的代码将公开提供。
%
%
面对如何有效利用复杂场景中丰富的光场线索的挑战,
本文提出了一种聚焦感知网络探索光场数据的方法。
%
%
该方法主要包含两个模块:令牌通信模块和焦点感知增强模块。
%
%
其中令牌通信模块通过嵌入式令牌表示汇总建立全聚焦图片和焦点堆栈的切片级特征,
并通过令牌作为信息传递的桥梁,促进网络对空间上下文建模。
%
%
焦点感知增强模块充分考虑不同聚焦切片对于显著性的影响,
通过判断每个散焦切片的聚焦程度,来突出不同焦点切片中
显著性区域,同时抑制非显著性区域带来的负面影响。
%
%
相比现有的方法,本文方法通过附加嵌入式令牌的方式,
对光场的整体三维场景进行了切片级的探索,
并考虑了不同散焦切片对显著性预测的贡献,
能够更有效的利用光场信息。

%%%%%%%%%%%%%%%%%%%%%%%%%%%%%%%%%%%%%%%%%%%%%
%
% 第二个工作点
%
%%%%%%%%%%%%%%%%%%%%%%%%%%%%%%%%%%%%%%%%%%%%%
(2)
%
%
面对如何高效的利用光场数据中全聚焦图和焦点堆栈两个模态的差异信息,
本文提出了一种视角增强网络探索光场数据的方法。
%
%
该方法主要包含两个主要部分:视角增强注意力模块和感知对比学习策略。
%
%
其中视角增强注意力模块通过对两个模态做交叉注意力时引入跨模态的掩码表达,
加强了注意力权重在不同聚焦区域上的显著性表达。
%
%
感知对比学习策略考虑显著性预测的前景区域内部,与背景区域内部的一致性表达。
%
%
相比现有的光场显著性检测方法,本文方法对光场数据进行跨模态的特征融合,
充分考虑了焦点堆栈和全聚焦图对最终显著性预测的贡献,
能够产生更为鲁棒的显著性物体表达。


%
%
%面对如何高效的挖掘有限光场数据的挑战,本文提出了一种区域感知网络探索
%光场数据的方法。该方法主要包含两个模块:多源学习模块和聚焦度识别模块。
%其中多源学习模块充分考虑各个焦点切片不同区域对预测的贡献,在显著性、边界以及中心位
%置等多个指导信息下生成区域级的注意力权重,突出不同切片中聚焦的显著性区域,并
%根据生成的注意力权重整合焦点堆栈的特征。聚焦度识别模块充分考虑多聚焦特性对显
%著性的影响,通过判断各个焦点切片不同区域的聚焦度以优化和更新注意力权重,使得
%特征整合过程中进一步突出显著性区域的同时抑制非显著性区域带来的影响。相比现有
%方法,本文方法对光场数据进行区域级的探索,充分考虑不同区域对最终预测的贡献,
%更有效的利用了光场信息。
%
%
\keywords{显著性目标检测;光场;聚焦感知;多模态特征融合; 交叉注意力}
\end{abstract}








%%==================================================
%% abstract.tex for DUT  Thesis
%% version: 0.1
%% last update: Apr 27th, 2022
%%==================================================

\begin{abstract}
%大连理工大学硕士研究生撰写学位论文应当符合写作规范和排版格式的要求,以下格式为研究生院依据国家标准和行业规范所编制的硕士学位论文模板,供硕士研究生参照使用。\par 论文摘要是学位论文的缩影,文字要简练、明确。内容要包括目的、方法、结果和结论。单位制一律换算成国际标准计量单位制,除特殊情况外,数字一律用阿拉伯数码。文中不允许出现插图,重要的表格可以写入。\par  摘要的主要内容为,简述全文的目的和意义、采用方法、主要研究内容和结论。\par 篇幅以一页为限,摘要正文后列出3-5个关键词,关键词与摘要之间空一行。\par“关键词:”是关键词部分的引导,不可省略。\par 关键词请尽量用《汉语主题词表》等词表提供的规范词。关键词之间用分号间隔,末尾不加标点。
%
%
%
%
	
	
显著性目标检测的目标在于识别图像中最引人注目的对象或区域,这是计算机视觉领域中一个重要的任务。现有的显著性目标检测算法根据输入数据的类型可以被分为三类:RGB、RGB-D和光场方法。与RGB和RGB-D数据相比,光场数据包含了更丰富的场景信息,可以满足对于复杂场景信息的需求。近年来,随着深度卷积神经网络的发展,它取代了传统的基于手工特征的算法,显著提升了光场显著性目标检测的性能。
%
%
%然而,实际应用中存在较高的光场数据获取成本、复杂的光场多线索信息处理以及耗时耗力的显著性像素级标注,这导致当前光场显著性目标检测数据稀缺,为深度模型提供足够支持的数据不足。为解决这些问题,本文从高效利用光场信息和增广光场数据两个角度出发,探索利用有限数据驱动的光场显著性目标检测方法。
%
%
然而,实际应用中存在复杂的光场信息提取以及跨模态的光场信息融合难等问题,
这导致了当前光场显著性检测深度模型难以有效辨别光场场景的的显著性物体表示。
%
%
为了解决这些问题,本文从焦点感知和视角增强两个角度出发,
探索基于聚焦感知的光场显著性检测方法。
本文的主要工作及创新点如下所述:


%%%%%%%%%%%%%%%%%%%%%%%%%%%%%%%%%%%%%%%%%%%%%
%
% 第一个工作点
%
%%%%%%%%%%%%%%%%%%%%%%%%%%%%%%%%%%%%%%%%%%%%%
(1)
%
%
%最近,光场显著性物体检测(LFSOD)因在复杂场景中利用丰富的光场线索取得显著改进而引起了越来越多的关注。虽然许多工作在这一领域取得了显著进展,但对其焦点特性的更深入洞察应该被开发。在这项工作中,我们提出了焦点感知变换器(FPT),可以高效地编码焦点堆栈内和全部焦点图像中的上下文。具体而言,我们引入了与焦点相关的令牌来总结图像特定特征,并且提出了令牌通信模块(TCM)来传达信息并促进空间上下文建模。通过精确编码的与焦点相关的令牌之间的信息交换,可以丰富每幅图像的特征并与其他图像相关联。我们还提出了焦点感知增强(FPE)策略,以帮助抑制嘈杂的背景信息。对四个广泛使用的基准数据集进行的大量实验证明,所提出的模型优于当前最先进的方法。我们的代码将公开提供。
%
%
面对如何有效利用复杂场景中丰富的光场线索的挑战,
本文提出了一种聚焦感知网络探索光场数据的方法。
%
%
该方法主要包含两个模块:令牌通信模块和焦点感知增强模块。
%
%
其中令牌通信模块通过嵌入式令牌表示汇总建立全聚焦图片和焦点堆栈的切片级特征,
并通过令牌作为信息传递的桥梁,促进网络对空间上下文建模。
%
%
焦点感知增强模块充分考虑不同聚焦切片对于显著性的影响,
通过判断每个散焦切片的聚焦程度,来突出不同焦点切片中
显著性区域,同时抑制非显著性区域带来的负面影响。
%
%
相比现有的方法,本文方法通过附加嵌入式令牌的方式,
对光场的整体三维场景进行了切片级的探索,
并考虑了不同散焦切片对显著性预测的贡献,
能够更有效的利用光场信息。

%%%%%%%%%%%%%%%%%%%%%%%%%%%%%%%%%%%%%%%%%%%%%
%
% 第二个工作点
%
%%%%%%%%%%%%%%%%%%%%%%%%%%%%%%%%%%%%%%%%%%%%%
(2)
%
%
面对如何高效的利用光场数据中全聚焦图和焦点堆栈两个模态的差异信息,
本文提出了一种视角增强网络探索光场数据的方法。
%
%
该方法主要包含两个主要部分:视角增强注意力模块和感知对比学习策略。
%
%
其中视角增强注意力模块通过对两个模态做交叉注意力时引入跨模态的掩码表达,
加强了注意力权重在不同聚焦区域上的显著性表达。
%
%
感知对比学习策略考虑显著性预测的前景区域内部,与背景区域内部的一致性表达。
%
%
相比现有的光场显著性检测方法,本文方法对光场数据进行跨模态的特征融合,
充分考虑了焦点堆栈和全聚焦图对最终显著性预测的贡献,
能够产生更为鲁棒的显著性物体表达。


%
%
%面对如何高效的挖掘有限光场数据的挑战,本文提出了一种区域感知网络探索
%光场数据的方法。该方法主要包含两个模块:多源学习模块和聚焦度识别模块。
%其中多源学习模块充分考虑各个焦点切片不同区域对预测的贡献,在显著性、边界以及中心位
%置等多个指导信息下生成区域级的注意力权重,突出不同切片中聚焦的显著性区域,并
%根据生成的注意力权重整合焦点堆栈的特征。聚焦度识别模块充分考虑多聚焦特性对显
%著性的影响,通过判断各个焦点切片不同区域的聚焦度以优化和更新注意力权重,使得
%特征整合过程中进一步突出显著性区域的同时抑制非显著性区域带来的影响。相比现有
%方法,本文方法对光场数据进行区域级的探索,充分考虑不同区域对最终预测的贡献,
%更有效的利用了光场信息。
%
%
\keywords{显著性目标检测;光场;聚焦感知;多模态特征融合; 交叉注意力}
\end{abstract}








%


\begin{englishabstract}
	
	Contents of the abstract.Times New Roman.
	
	\englishkeywords{Write Criterion; Typeset Format; Master's Degree Thesis}
	
\end{englishabstract}



%% 符号对照表,可选,如不用可注释掉

% 加入目录
%\tableofcontents
%\tableofengcontents

% 加入图、表索引(同时取消图表索引中章之间的垂直间隔)
%\let\origaddvspace\addvspace
%\renewcommand{\addvspace}[1]{}
%\addcontentsline{toc}{chapter}{图目录}
%\listoffigures
%\addcontentsline{toc}{chapter}{表目录}
%\listoftables
%\renewcommand{\addvspace}[1]{\origaddvspace{#1}}

% 主要符号表
%\input{chapters/denotation}

%%%%%%%%%%%%%%%%%%%%%%%%%%%%%%
%% 正主体部分
%%%%%%%%%%%%%%%%%%%%%%%%%%%%%%
\mainmatter

%% 各章正文内容
\let\cleardoublepage\clearpage

%%%==================================================
%% chapter01.tex for DUT Thesis 
%% version: 0.1
%% last update: Dec 25th, 2022
%%==================================================
\BiChapter{绪论}{Introduction}
绪论应包括本研究课题的学术背景及其理论与实际意义;本领域的国内外研究进展及成果、存在不足或有待深入研究的问题;本研究课题的来源及主要研究内容等。
\label{chap:part1}
\BiSection{研究背景与意义}{Research background}

随着互联网技术的迅猛发展,当今社会已迈入信息化时代。信息以各种形式呈现,包括文字、图像、音频和视频等。图像作为生动的信息表达方式在现实世界中起着重要作用。随着图像获取设备的不断进步,图像数据量急剧增长。研究者们希望利用计算机处理这些海量数据,以降低人工处理成本,形成了计算机视觉这一重要研究领域,吸引着大量研究者的探索。

在处理图像信息时,并非所有信息都是有效的。对图像全面处理会耗费大量计算资源,因此研究者们希望计算机能够像人类视觉系统一样,聚焦于主要信息,减轻计算负担。人类视觉系统能够快速定位到场景中感兴趣的目标区域,然后进一步加工处理,这被称为视觉注意力机制。这种机制使我们能够快速捕捉到场景的重要内容,提高信息处理效率。随着网络技术和图像采集设备的普及,我们每天接收大量图像数据。研究者期望计算机具备人类视觉注意力机制的能力,以提高图像数据处理效率,减少计算资源浪费。因此,显著性目标检测任务应运而生,旨在模拟人类视觉注意力机制,识别显著性目标或区域。

%这一任务在计算机视觉、计算机图形学和机器人技术等多个领域发挥着重要作用,为其他视觉任务提供有效帮助,并在各种任务中应用广泛。

显著性目标检测在计算机视觉、计算机图形学和机器人技术等领域扮演着关键角色。在计算机视觉领域中,显著性目标检测为其他视觉任务提供重要支持,在语义分割、目标检测以及目标追踪等任务中有广泛应用。在计算机图形学领域,它被运用于自动图像裁剪、图像重定位和视频摘要等任务。在机器人技术中,显著性目标检测被用于辅助人机交互和目标识别等任务。

当前的显著性目标检测方法基于输入数据类型不同,可分为三类:2D数据包括RGB图像,3D数据包括RGB-D图像,以及4D光场数据。随着深度卷积神经网络的进步,显著性目标检测已经从传统手工特征转向主动探索图像语义特征,性能得到明显提升。基于深度学习的2D显著性目标检测方法利用CNN挖掘RGB图像的特征,进而预测显著性图。然而,在面对一些挑战性场景时,由于缺乏足够的场景信息支持,2D方法可能产生错误预测。

相较于2D数据,3D数据包含更多空间信息,并且随着3D传感技术的发展而更易获取。基于深度学习的RGB-D显著性目标检测方法致力于融合深度特征和RGB特征,在推理显著图时考虑场景深度信息。许多RGB-D显著性目标检测方法因此获得较高检测性能,尤其在挑战性场景中表现突出。然而,3D方法受深度图质量影响,低质量深度图可能导致错误的预测,这是深度传感器采集问题带来的挑战。

相对于常规相机拍摄的RGB图像和RGB-D数据,光场相机捕捉的光场数据记录了更加全面和详尽的自然场景信息,包含深度线索、焦点线索和角度信息等。焦点堆栈数据是光场数据的一种表达方式,表示为一组在不同深度范围内聚焦的图像。现有研究表明,利用焦点堆栈数据的光场显著性目标检测方法在处理前景背景相似、复杂背景和小物体等具有挑战性场景时具有显著优势。然而,在实际场景中,光场数据获取成本高、处理多线索信息的成本高以及显著性像素级标注成本高,导致当前光场数据库样本数量较少。由于数据驱动的深度模型受数据限制,因此相同模型在训练数据较少的情况下会表现出较差的检测性能。因此,需要有效利用有限光场数据的网络模型,并通过增强当前光场数据集的算法来缓解这些问题。

因此,这篇文章将以有限数据驱动的光场显著性目标检测为研究重点,从网络模型和数据增强两个方面进行探索。首先,提出了区域感知网络,将焦点堆栈特征从全局整合到局部,更有效地利用现有光场数据。此外,还提出了一种基于数据增强的光场显著性目标检测方法,通过在重新组合显著性目标和背景的基础上合成新数据,从而扩充现有光场数据集。


\BiSection{国内外相关研究工作进展}{Research progress}

近年来,显著性目标检测作为计算机视觉领域的重要基础任务之一,已经成为研究的焦点之一。研究学者们提出了许多出色的检测模型,涵盖了RGB显著性目标检测、RGB-D显著性目标检测和光场显著性目标检测这三种类型。以下概述了近年来国内外该领域的研究进展。


\BiSubsection{RGB 显著性目标检测}{TODO}

目前的RGB显著性目标检测方法大致可以分为两类:传统方法和基于深度学习的方法。传统方法属于自底向上的方法,主要依赖各种颜色、纹理等低阶手工特征或对比度、中心环绕等启发式先验线索来识别显著性目标。这些传统方法的主要优势在于快速和适应性强,其中常见的显著性模型包括对比度先验、中心先验和对象性先验。对比度先验可分为局部对比度和全局对比度方法,前者主要通过像素或区域之间的对比度来定位显著性,而后者则考虑整个图像不同区域或像素之间的对比度。中心先验假设显著性对象通常位于图像中心,可用于优化显著性预测图。例如,经过全局对比模型生成初始显著性图后,可以利用中心先验模型进一步提升图像质量。对象性先验常用于改善粗略显著性图的性能,通过比较不同区域的特征来估计显著性图。


以上提及的方法利用确定性先验知识,在前景和背景对比度较高的场景中具有较高的检测准确性,但它们的泛化能力较差,对于不符合这些先验知识的场景,检测精度会明显下降。基于深度学习的方法属于自顶向下的方法,是一种数据驱动的方法,主要通过训练经过设计的网络模型来定位显著性目标,利用大规模带标签的数据库。由于卷积神经网络的发展,基于深度学习的检测方法的性能显著提高。一些方法集中在网络结构的设计上:DSS提出了多层特征密集连接的解码网络结构,结合不同尺度的特征以提高检测性能;R3Net提出了多层循环网络结构,通过迭代方式融合高阶和低阶特征,逐步优化预测结果;CPD设计了部分解码网络结构以提高检测速度而不影响性能。随着注意力机制的发展,一些方法利用注意力机制挖掘场景中的有效特征。例如,PiCANet结合了空间和通道注意力,将注意力模块嵌入多路循环网络以提高检测性能;PAGRN设计了渐进式的注意力引导模块,有选择性地整合多层特征的上下文信息。近年来,为了提高检测性能,一些显著性目标检测方法开始关注分割边界的性能,如将IOU损失用作惩罚损失来提高预测图的边界性能;另外,一些方法在网络中明确地对互补的显著性目标信息和显著边缘信息进行建模,以维持显著目标边界的预测精度。


\BiSubsection{RGB-D 显著性目标检测}{TODO}


在RGB-D的显著性目标检测领域,深度图提供丰富的空间信息,这为在许多复杂场景下的显著性目标检测性能带来了显著提升。传统的RGB-D检测方法通常依赖形状、三维布局等低阶特征来定位显著性目标。Peng等学者提出了多阶段的检测模型,结合深度信息和图像信息进行显著性预测。Ren等研究者将RGB-D视为4通道数据来计算局部对比度,并将对比信息与全局先验知识相结合,提出了基于两阶段的显著性目标检测框架。一些RGB-D检测方法基于特定假设来定位显著性目标,比如认为靠近相机的目标更容易被识别为显著性目标。Feng等学者根据这一假设,提出了一种局部背景封闭特征来识别显著性目标。这些方法证明了深度信息在显著性目标检测任务中的重要作用。然而,与传统RGB检测方法类似,这些方法存在泛化性较差的问题,当不满足先验知识或假设时,检测精度会明显下降。随着深度学习的进展,卷积神经网络能够提取深度图和RGB图像的高阶语义特征,从而大幅提升了RGB-D检测方法的性能水平。


基于深度学习的 RGB-D 检测方法通常先提取RGB-D数据的特征,然后融合不同模态的特征来定位显著性目标,许多研究致力于研究更有效的跨模态特征融合方式。根据融合方式的不同,现有的基于深度学习的 RGB-D 方法可以大致分为前期融合、后期融合和多级融合三种方式。前期融合策略是指先融合多模态信息,然后提取特征来预测显著性,Qu等研究者遵循这种方式,将RGB图像和深度图同时输入深度网络进行显著性预测。后期融合策略则是先提取多模态信息特征,然后融合这些特征来预测显著性,Shigematsu等学者采用后期融合方式,分别提取RGB图像和深度图的特征,然后将它们级联来定位显著性目标。与前述两种方式相比,多级特征融合在不同层级进行跨模态特征融合,使得不同层级的特征相互补充,在显著性目标检测任务中更为有效。Chen等研究者在PCA中采用多级特征融合方式,引入跨模态互补感知融合模块,考虑RGB和深度图之间的联系,在特征融合时获取更充分和有效的信息。Piao等研究者提出了深度诱导的多尺度注意力网络,结合深度特征和RGB特征以不同尺度上下文信息融合,实现了精准的显著性定位。此外,研究者们还积极探索除特征融合外的RGB-D检测方法。在CPFP中,Zhao等学者利用增强的深度信息来增强显著性目标与背景之间的对比度,并将其与RGB特征级联以预测显著性。Chen等研究者提出了一个自适应整合深度特征和RGB图像特征的注意力机制。


\BiSubsection{光场显著性目标检测}{TODO}


在光场显著性目标检测领域,光场数据能够记录目标场景的空间信息并提供准确的深度信息,从而缓解困难场景下检测准确性受限的问题。光场数据记录了自然场景更全面、更完整的信息,对于显著性目标检测任务具有积极作用,因此越来越多关于利用光场数据提升检测性能的研究开始涌现。传统光场显著性目标检测模型通常基于各种手工特征或先验假设来定位显著性目标,比如色彩对比度、背景先验等。大多数方法采用先进行粗糙预测,然后在细化处理的多阶段检测方式,而且大部分研究都是基于光场焦点堆栈数据展开探索。Li等研究者在 LFS 中首次在光场数据上实施了显著性目标检测,并构建了首个光场显著性检测数据集。LFS 结合了焦点聚焦和位置先验来确定背景与前景的焦点切片,然后通过背景先验计算出的显著性图与选定的焦点切片相结合来定位显著性目标。在WSC中,Li等人提出了一个可以处理2D、3D和4D光场数据的框架,利用非显著性字典对图像进行重构,并以高重构误差区域作为显著性字典,并通过迭代细化检测显著性目标。DILF 首先对光场数据进行超像素分割,然后通过计算深度线索对比度和RGB图像对比度生成显著性图,并利用背景先验增强显著性目标。Wang等学者提出了一个基于贝叶斯的显著性目标检测框架,能够有效整合从光场中提取的各种视觉特征。Zhang等研究者基于随机搜索策略,整合了全聚焦图像、深度图、焦点切片和多视角图像中的光场线索。与RGB方法以及RGB-D方法类似,传统的光场检测方法由于手工特征的泛化性较差,很难推广到较为困难的场景。



基于深度学习的光场显著性目标检测方法可以根据输入数据分为两类:一种是利用焦点堆栈数据的检测方法,另一种是利用多视角图像的检测方法。大部分基于深度学习的光场方法都专注于利用焦点堆栈来检测显著性目标,致力于寻找更有效的多模态特征融合手段。在DFS中,Wang等研究者利用ConvLSTM生成注意力向量,用于加权焦点切片特征和全聚焦图像特征,整合光场数据特征并预测显著性值。Piao等团队也运用ConvLSTM来探索光场数据,并创建了目前规模最大的光场显著性目标检测数据集。在LFNet中,Zhang等人提出了细化模块和整合模块,通过整合模块融合光场特征,并利用细化模块进一步优化显著性预测值。在PANet中,Piao等研究者提出了一种区域级的探索光场数据方法,并提出了相应的特征整合策略,利用光场的全局信息来缓解大物体多目标检测不准确的问题。另一方面,利用多视角图像进行显著性检测的深度学习模型致力于从不同视角间挖掘有效特征之间的关联。在DLSD中,Piao等学者将光场显著性目标检测拆分为两个子任务:通过合成多视角图像来检测显著性对象。而Zhang等研究者在MAC中采用多视角图像阵列进行光场显著性目标检测,并建立了一个新的多视角图像阵列的数据集,通过对不同视角图像角度变化建模,将提取的特征输入至DeepLabV2的结构中以捕获多尺度信息。


\BiSubsection{现存问题}{TODO}

由于深度学习的快速发展,网络模型能够提取出光场数据的高级语义特征,这导致光场显著性目标检测取得了显著的进展。对于跨模态特征融合,目前使用焦点堆栈数据的深度模型可以有效地融合多模态特征,并充分挖掘场景的几何关系。然而,深度学习方法是数据驱动的,因此用于光场显著性目标检测的深度模型性能受到数据量的影响。一般情况下,训练样本越多,模型的泛化能力就越好,检测效果也越强。不过,在实际场景中,获取光场数据需要多组摄像头或相机矩阵完成,这带来了极高的采集成本。而由于光场数据包含多个线索,如焦点堆栈、深度信息和多视角信息,处理这些数据的成本也很高。此外,为了进行显著性目标检测,需要对光场数据进行像素级标注,这需要耗费大量时间和精力。

这些问题导致现有光场显著性目标检测数据集非常稀缺,难以为现有深度学习算法提供足够的支持。因此,我们需要有效利用有限的光场数据来设计网络模型,并采用能够对有限光场数据进行增强的算法来解决上述问题。然而,当前的光场显著性目标检测方法通常在网络模型设计上引入全局注意力模块,专注于关注深度范围内与显著性目标距离较近的焦点切片特征,却忽略了焦点堆栈中其他切片的特征,导致光场中大部分信息未得到充分利用。同时,现有的数据增强方法通常使用传统的方式,如图像翻转、旋转、裁剪和颜色变换等,这些方法基于先验知识,无法覆盖所有可能的测试场景。

综上所述,基于有限数据的光场显著性目标检测研究面临着挑战,并需要进一步深入研究。


\BiSection{论文主要内容及结构安排}{TODO}
\BiSubsection{主要内容}{TODO}

本篇文章以显著性目标检测为核心内容,着眼于探索受限数据驱动的光场显著性检测方法。研究从两个方面入手,一是设计能有效利用有限光场数据的网络模型,二是构建有效增强光场数据的算法,以缓解数据稀缺的挑战。在关于如何设计适用于有限光场数据的深度检测网络模型方面,本文提出了一种新的区域感知网络。这个网络与目前方法中使用的全局注意力不同,它从局部角度出发,考虑了每个焦点切片中不同区域对显著性预测的作用,更充分地利用了有限的光场数据。多源学习模块结合显著性、边界和中心位置信息生成特征整合策略,针对焦点堆栈的特征进行区域级整合,聚焦性识别模块考虑多聚焦特性对显著性的影响,并更新整合策略以更好地突出显著性区域并抑制非显著性区域。

另一方面,关于如何设计有效增强光场数据的算法,本文提出了一种基于数据增强的光场显著性目标检测方法。相对于传统的数据增强方法,该方法引入了几何增强模块,通过结合图像修复网络和空间变换网络重新组合场景中的显著对象和背景,以尽可能扩增当前的光场数据集。聚焦性补偿模块则利用风格迁移网络进一步优化组合图像中焦点堆栈的真实性。此外,该方法还提出了一个不确定性学习策略,用于联合训练合成数据和真实数据,通过不同对待质量的合成数据,减小合成数据对网络训练的不利影响。

\BiSubsection{结构安排}{TODO}


第一章探讨了聚焦感知的光场显著性目标检测方法的背景和意义,以及对国内外显著性目标检测领域的研究现状与进展作了介绍。分析了目前方法中存在的问题与不足,并总结了本文的解决思路。

第二章详细介绍了相关理论知识,包括光场和光场显著性目标检测的理论基础。

第三章详细介绍了基于焦点感知的光场显著性目标检测方法。首先分析了现有光场显著性目标检测方法存在的问题,并阐述了研究的动机。随后介绍了所提出的网络模型,以及涉及到的多源学习模块、聚焦度识别模块和特征整合策略的具体实施方法。最后,展示了实验设置和结果分析,证明了所提出方法的卓越性能。

在第四章中,详细阐述了基于视角增强的光场显著性目标检测方法。探讨了增强光场数据的研究动机,并详细介绍了几何增强模块、聚焦性补偿模块和不确定性学习策略的具体实施方法。最后,呈现了实验设置和结果分析,以验证所提出方法的显著优势。


%
\BiChapter{相关理论}{TODO}
\label{chap:part2}

\BiSection{光场技术基本理论}{TODO}

光场的概念由A.Gershun~\cite{gershun1939light}教授提出,是空间中光线集合的完备表示,采集并显示光场就能在视觉上重现真实世界。
1991年,MIT的Edward H.Adelson教授和James R.Bergen\cite{adelson1991plenoptic}教授指出人眼对光线的视觉感知可以认为是沿着单一函数的一个或多个方向的局部变化,描述了光照射到观察面的信息结构。
一旦定义了这个函数,各种潜在的视觉属性(如运动、颜色和方向)的测量就能够自动分离出来。
这个函数被称为全光函数,表示为:
%$$$$
\begin{equation}
	L(x,y,z,\theta,\varphi,\lambda,t)
\end{equation}\par
其中$(x,y,z)$为发光物体的空间位置,$(\theta,\varphi)$分别表示传播光线入射的垂直角度和水平角度,$\lambda$表示传播光线的波长,发光物体所发射的光线信息随时时间$t$的推移而变化。


然而,这种能够记录空间中光线信息的七维全光函数过于复杂、数据量大,难以记录和存储,在实际计算中并未得到应用。
需要对其进行简化处理。
McMillan等~\cite{mcmillan2023plenoptic}在七维全光函数的基础上提出了简化波长和时间的更为方便的五维光场模型。
通过记录红、绿、蓝三原色来简化波长$\lambda$,通过记录不同帧来简化时间$t$。
如果不考虑光线在空间传播过程中的衰减,记录光场信息的五维模型还可以进一步简化成四参数光场函数。
Levoy等~\cite{levoy2023light}忽略掉传播距离维度$z$得到四维光场模型,
发展出了适用于光学系统的光场双平面参数特征。
\begin{figure}[!ht]
	\centering
	\includegraphics[width=0.78\linewidth]{figures/chapter2/double-plane2}
	\bicaption{光场双平面四参数模型}{Light field biplane four-parameter model}  
	\label{chapter2_fig1:double_plane}
\end{figure}
光线在空间传播中,因传播距离而造成的信息损耗微乎其微,光场模型完全可以以简化的四参数模型表示。




% \emph{et~al.}~
%光场是计算机科学领域的学者定义的“Light Field”,是指除了包含原图像矩阵中的空间坐标$(x,y)$和强度$I$外,还有光线入射的角度信息$(\theta,\varphi)$。
%光在传播过程中的各种潜在的视觉属性(如运动、颜色和方向)。


\BiSubsection{光场成像原理}{TODO}
\BiSubsection{光场数据可视化}{TODO}

\BiSection{光场显著性目标检测相关理论}{TODO}

\BiSubsection{基于多视角图像的显著性目标检测原理}{TODO}
\BiSubsection{基于焦点堆栈的显著性目标检测原理}{TODO}
\BiSubsection{显著性目标检测性能评估指标}{TODO}


\BiSection{本章小节}{TODO}

本章首先阐述了光场技术的基本理论,介绍了光场的成像原理以及数据可视化形式;
然后介绍了光场显著性目标检测的相关理论,分别描述了基于多视角图像的显著性目标检测、
基于焦点堆栈的显著性目标检测方法的实现原理,
并引入了显著性目标检测中常用的几种性能评价指标。
%
\BiChapter{基于焦点感知的光场显著性目标检测}{TODO}
\label{chap:part3}

\BiSection{研究动机}{TODO}
%
%
%
%
光场显着物体检测(LFSOD)由于光场中包含丰富的空间信息而引起了广泛的关注。
与 2D (RGB) 和 3D (RGBD) 数据不同,光场本质上捕获结构化 4D 表示,包括多视图图像、深度图和焦点切片。 其中,通过眼球运动顺序观察切片的焦点堆栈,
以及迎合人类视觉感知的可见注意力转移~\cite{piao2020dut},做出适合显着性对象检测。

一些开创性的方法~\cite{zhang2019memory,piao2020exploit}~采用 ConvLSTM~\cite{shi2015convolutional}~,它使用记忆机制以预定义的顺序单独处理焦点堆栈特征。张等人~\cite{zhang2021learning}~后来在编码器阶段采用3D卷积来提取特征。刘等人~\cite{liu2021light}~和张等人~\cite{zhang2021geometry}~使用图神经网络来聚合不同焦点切片中的上下文信息。 这些方法依赖内存使用或大量计算来提取焦点堆栈特征,从而限制了效率。
我们重新思考光场数据建模的方式。考虑到焦点堆栈的成像效果,如图1所示,每个焦点切片根据空间透视深度的不同,聚焦部位也不同。 并且从同一场景生成,焦点切片有很多共同点。

因此,即使在很小的带宽内,也可以充分总结和传达它们之间的差异。 此外,给定图像,人类可以毫不费力地关注敏感部分并忽略不重要的背景。 因此,在焦点堆栈中处理更多相对的焦点切片来模拟人类视觉系统是合理的。 受上述观察的启发,我们考虑两个关键问题:

1)我们如何设计一个模型来存储切片级特征并在焦点堆栈和全焦点图像之间传递信息以进行上下文建模? 
2)我们如何设计一个模型来理解场景的空间分布并感知敏感的焦点切片? 
%
%
%
%
\par
在本文中,我们提出了一种用于高效且有效的 LFSOD 的焦点感知Transformer(FPT)。 具体来说,我们收集图像特定的特征并制定通信以加强焦点堆栈和全焦点图像之间的相互感知。 我们利用选择性机制将适当的焦点切片纳入检测。 具体来说,我们的贡献有三个:
%
%
%
%
\par
%
%
%
%
\begin{itemize}
	\item 我们引入与焦点相关的标记来总结图像特定的特征,并提出一种标记通信模块(TCM),通过计算与焦点相关的标记之间的交叉注意力来执行特征交互。 我们转移焦点堆栈中与焦点相关的标记以促进空间上下文传播。 
	
	\item 我们提出了一种焦点感知增强(FPE)策略,通过切片选择机制来增强焦点堆栈中的特征空间表示,以有区别地处理适当的焦点切片。	这可以突出显着切片并抑制非显着区域的干扰。 
	
	\item 我们对 4 个广泛使用的数据集进行了广泛的实验,并证明我们的方法优于现有最先进的 LFSOD 方法。 我们的方法在 DUTLF-FS~\cite{zhang2019memory}~上将 MAE 指标显着降低了 31\%。
\end{itemize}


 RBG显著性目标检测(SOD)~\cite{ ma2021pyramidal, wei2020f3net, zhou2020interactive}~、
 RGB-D~\cite{ cong2022cir, ji2021calibrated, liu2021visual}~和光场图像一直是活跃的研究领域。 
 在本文中,我们将主要研究LFSOD任务。
 现有的光场显着目标检测方法大致可以分为两类:(1)传统方法;(2)基于深度学习的方法。 传统方法通常采用手工制作的特征(例如,颜色对比度、纹理对比度和深度对比度)和先验(例如,位置先验、背景先验和边界连接先验)来检测显着对象。 
% 
%
%
%
 李等人~\cite{li2014saliency}~提出了第一个光场显着性数据集,并通过计算背景先验、位置先验和对比度线索来检测显着对象。
 之后,李等人~\cite{li2015weighted}~提出了加权稀疏编码框架同时处理2D、3D和4D SOD问题。
 张等人~\cite{zhang2015saliency}~计算对比度显着图,然后使用背景先验来消除背景干扰并获得最终结果。 
 张等人~\cite{zhang2017saliency}~基于随机搜索策略集成了从全焦点图像、深度图、焦点切片和多视图图像中提取的多个光场线索。 
 最近,Piao 等人~\cite{piao2019saliency}~提出了 LFSOD 的深度诱导元胞自动机。
 有关传统方法的更多详细信息可以在\cite{fu2022light}~中找到。
 到了深度学习时代,几种深度学习方法对光场SOD性能有了显着提升。 
 朴等人~\cite{piao2019deep}~首次尝试引入CNN来提取光场语义特征,并获得相应的显着图。 
 王等人~\cite{wang2019deep}~应用ConvLSTM~\cite{shi2015convolutional}~来融合CNN生成的特征,然后预测显着图。
 张等人~\cite{zhang2019memory}~还利用ConvLSTM来利用光场,并提出了用于显着性检测的最大光场数据集。 
 张等人~\cite{zhang2020lfnet}~利用提出的细化模块和集成模块的集成焦点堆栈特征来开发焦点切片。
朴等人~\cite{piao2020exploit}~提出了一种由焦点流和RGB流组成的不对称双流架构,以实现台式计算机和移动设备的多功能性。
%
%
%
%
\par
%
尽管大多数方法输入全焦点图像和焦点堆栈,但一些方法\cite{jing2021occlusion, wang2022lfbcnet, zhang2022exploring}~提出使用多视图和中心视图图像来检测显着对象。 
张等人~\cite{zhang2020light}~提出了一种深度网络,通过利用微透镜图像中丰富的角度信息来检测显着物体。 
张等人~\cite{zhang2021geometry}~提出了一种图神经网络,通过有效探索多视图图像之间的空间和视差相关性来预测显着图。 
静等人~\cite{jing2021occlusion}~提出了一种遮挡感知网络,从极平面图像(EPI)中提取遮挡边界特征以进行显着性检测。 
张等人~\cite{zhang2022exploring}~提出了一种光场合成网络来产生可靠的4D信息并驱动显着性检测。 
然而,上述方法的性能不如基于焦点堆栈输入的常见方法。 

这些使用焦点堆栈作为输入的方法集成了解码器中整个焦点堆栈的特征,忽略了不同切片对检测的相对贡献,并且容易受到非显着背景的影响。 因此,我们提出了更为适合 LFSOD 任务的设计。
 

Transformer,首先由 Vaswani 等人提出~\cite{vaswani2017attention}~,已广泛应用于自然语言处理(NLP)。 ViT 由 Dosovitskiy 等人提出。 [7]首先将Transformer应用于图像域。 由于其强大的全局信息捕获能力,变压器表现出了优异的性能。
最近的工作探索了将 Transformer 应用于各种视觉任务:图像分类 [3, 7]、对象检测 [60, 5, 35]、分割 [2, 41]、图像增强 [45, 2]、图像生成 [25] 和 视频处理 [59, 57],以缓解 CNN 有限的全局信息学习能力。 此外,Transformer 已广泛应用于 RGB 显着目标检测 [20, 34] 和 RGBD 显着目标检测 [22, 40],例如,Liu 等人。 [20]设计了一个基于纯变压器架构的统一模型,通过建模远程依赖性来预测显着性。 刘等人。 [22]提出了一种用于 RGB-D 显着目标检测的三元组变换器嵌入模块,通过学习跨层的远程依赖关系来增强高级特征。 塞里斯等人。 [34]提出了一种上下文实例转换器来捕获对象和场景上下文之间的上下文关系,以实现更准确的显着性推断。 王等人。 [40]提出了一种基于变压器的多模态融合模块来增强和融合RGB和深度图像特征。 受益于变压器的使用,这些方法可以获得更准确的场景上下文特征,并在复杂场景中表现出更好的检测性能。 然而,如何将变压器应用于 LFSOD 尚未得到全面探讨。 考虑变压器在建立长期依赖方面的优势,适合总结图像特定特征并通过附加标记传达信息。 在本文中,我们提出了一种令牌通信模块(TCM)来加强信息交互并促进上下文特征感知。


\BiSection{方法介绍}{TODO}

我们提出的 FPT 的整体架构如图 2 所示。给定 N + 1 个分辨率为 H × W 的图像,包括一个全焦点图像和 N 个焦点堆栈图像,我们将每个图像划分为补丁作为输入。 将展平的 { }N 个 patch 输入到线性投影中,得到嵌入的 patch fi 0 i=0,其形状为 (N + 1) × HW P 2 × C,其中 P 表示每个 patch 的大小,C 表示每个 patch 的大小。 渠道维度。 具体地,f0 0 表示全焦点图像的{ }N个块。 fi 0 i=1 表示焦点堆栈的斑块。 此外,为了总结补丁中的信息,引入了一组大小为 M × C 的随机初始化的可学习嵌入,作为 { }N 个焦点相关标记,表示为 m0 ,其中 M 表示焦点相关标记的数量。 我们将 i i=0 {[ ]}N 每个 patch 与焦点相关的标记连接起来,并得到 fi 0 , m0 i i=0 作为特征提取器的输入。 我们基于金字塔视觉变换器(PVT)[39]的主干网由 T = 4 个阶段组成,每个阶段包含 Ni ∈ [3,4,6,3] 变换器块。 如图 2 所示,每个 Transformer 块包括线性空间减少注意力 (SRA) 和前馈网络 (FFN),以每个图像的方式作用于联合标记:其中 l ∈ 1..T 表示阶段编号 。 每个变压器编码器后面都布置了令牌通信模块(TCM),用于特征通信。 { }N 特征提取后,我们可以从 Transformer 块的输出中得到四层特征 fi 1, fi 2 , fi 3 , fi 4 i=0。 然后,采用共享权重特征金字塔网络(FPN)[18]进行分层融合并得到 { }N 个金字塔特征图 Fi 1, Fi 2, Fi 3 , Fi 4 i=0:其中 U p 贡献 2× 上采样 操作时,CBR 捐赠一个卷积块,包括 Conv+ BN+ RELU 层。 对于解码器,提出的焦点感知增强(FPE)策略增强了焦点堆栈的特征表示。 增强的焦点堆栈和全焦点特征融合在特征融合模块(FFM)中。 最后,采用掩模解码器来生成显着图。

\BiSubsection{令牌交互模块}{TODO}

在以前的 LFSOD 方法中,主干网络仅以原像方式执行特征提取 [31, 21],忽略了光场数据固有的丰富上下文信息。 相比之下,我们设计了一个令牌通信模块(TCM)来对全焦点和焦点堆栈进行上下文建模,如图 3 所示。TCM 由令牌交互(TI)和令牌移位(TS)操作组成。 TC 计算机在全焦点和焦点堆栈的焦点相关令牌之间进行交叉注意,以进行信息交互。 TS 转移焦点堆栈中与焦点相关的标记,以促进上下文特征感知。 代币互动。 如图 3 (a) 所示,TC 由交叉注意力 (CA) 块组成。 注意力函数可以描述为将查询和一组键值对映射到计算为值的加权和的输出。 分配给每个值的权重是查询与相应键之间的相似度。 这里,我们采用带有残差连接和层归一化的缩放点积注意力来实现 CA 块,其可以表述如下: √ 其中 Q ∈ RNq ×dk 、 K ∈ RNk ×dk 和 V ∈ RNv ×dk 是 分别是查询、键和值。 dk是查询和密钥的通道维度,dk是控制softmax分布的温度参数。 对于CA块,K和V是相同的。 { }N { }N CA 块将所有与焦点相关的标记 m1 i i=0 作为输入。 焦点堆栈流的标记mi 1 i=1用作查询来计算与属于全焦点流的键m10的相似度并从值m10检索焦点信息。 该 CA 块的输出 Q1 可计算如下:其中 Q = m1 i i=1 W Q 、K = m10 W K 、V = m10W V 。 Q、K、V 和 Qlout 的维度分别为 N × M × C、M × C、M × C 和 N × M × C。 令牌移位。 上述操作的输出 Q1 被馈送到令牌移位操作中,如图 3 (b) 所示。 与焦点相关的标记被分为 G = 2 组,并沿着焦点深度轴以不同方向(向前或向后)循环移动。 通过不同的焦点深度和方向,令牌可以实现与近深度和远深度焦点图像的空间{}N上下文交换。 然后我们用连接的 m10 更新所有与焦点相关的标记 m1i i=0 并将 Q1 移出。 这种设计的目的是随着网络的深入,保持稳定的空间感受野。 值得一提的是,TS几乎是无参数的,带来的计算成本可以忽略不计。


\BiSubsection{聚焦感知增强策略}{TODO}


受人类视觉注意系统中关注选择性的启发,我们的目标是从多焦点特征中有效地感知有用的显着性信息。 我们提出了一种焦点感知增强(FPE)策略来模仿人类如何从视觉资源中选择感兴趣信息的筛选阶段。 FPE 由切片选择机制组成,用于有区别地处理相关焦点切片,如图 2 所示。选择机制可以突出显着切片并抑制非显着区域的干扰。 此外,我们采用结构相似性[42]来评估焦点切片和全焦点图像之间焦点的一致性,因为散焦状态下的物体不具有清晰的纹理结构。 首先,我们使用基于 FPN 的辅助解码器(AD)生成辅助显着性预测 P A 并应用监督以避免非显着对象的干扰并确保全焦点特征的有效性。 然后,我们使用全焦点显着性预测 P A 计算焦点堆栈 Fi l i=1 的每个层特征的结构相似度得分。 其中 CBR 捐赠了一个卷积块,它将特征压缩到一个通道。 SSIM表示结构相似度,可以表示为: 其中x和y捐赠两个输入图像,μx、μy和σx,σy分别是x和y的均值和标准差,σxy是它们的协方差,C1 = 0.012并且 C2 = 0.032 用于避免被零除。 生成scoreli后,我们选择前K个对应的特征作为增强数据:
其中T opK 是一种选择机制,在不改变原始空间顺序的情况下,根据scoreli 选择K 个最大值。 这种选择性焦点感知增强策略强调显着特征,同时抑制不必要的特征,这对于准确的显着目标检测至关重要。


\BiSubsection{训练过程}{TODO}
\todo 

在获得增强的焦点堆栈和全焦点特征后,逐步设计特征融合模块(FFM)来融合特征,如图2所示。与全焦点特征相比,焦点堆栈特征通常具有更高的数据维度, 一般为12倍大。 因此,平衡差异化的数据维度可以被认为是特征函数的前提任务。
一些简单的解决方案直接连接高维特征并使用 2D 卷积来压缩特征 [27] 或对每个焦点切片特征采用元素方式添加到融合数据 [21]。 然而,这些方法可能会阻止它们完全提取空间上下文信息,因为焦点堆栈在空间维度上是对齐的。 换句话说,生成的低维焦点堆栈特征无法提供足够的指导来实现高预测精度。 我们提出了一种基于 3D 卷积的通道压缩模块 (CCM) 来解决这个问题。 CCM引入了3D卷积,它可以本质上提取时空特征来融合所有焦点堆栈特征。
具体来说,对于形状为 k × C × h × w 的每一层增强焦点堆栈特征,我们将其重塑为 C × k × h × w,然后使用内核为 k × 1 × 1 的 3D 卷积来融合空间上下文特征 并压缩通道。 压缩特征C l 可以表示为: 其中CCM是3D卷积块,包括3D Conv+BN+RELU。 给定与输入形状相同的焦点堆栈和全焦点特征,采用多个卷积块进行跨域融合。 因此,我们将FFM表述如下:其中CBR 1和CBR2表示卷积块,包括Conv+BN+RELU,前者使用1×C通道输入,后者使用2×C。得到四层输出特征Ol后, 基于 FPN 的解码器 [18] 生成四个显着性 l=0 预测图。 最后一张用作最终的显着性预测图。 二元交叉熵(BCE)[6]是二元分类和分割中使用最广泛的损失函数。 但 Lbce 是像素级损失,这意味着它平等对待所有像素。

在具有主导背景的图片中,前景像素的损失将会被稀释。 最近,SOD[32]中引入了交集交集(IoU)来弥补 BCE 的不足。
Liou的目标是优化全局结构,而不是专注于单个像素,这样就不会受到分布不平衡的影响。 增强对齐度量[9]首先被提出作为一种可以同时考虑像素级和图像级误差的评估指标,损失形式为Lem = 1 − EΨ 。 基于上述讨论,我们构建了一个混合损失:如图 2 所示,我们的模型有两个掩码标头,它们预测辅助显着性图 PA 和 { }3 四个最终多尺度显着性图 Pi S i=0 。 因此,所提出的网络的总损失 Ltotal 可以表示为: 其中 G 表示地面实况显着性图。 L 代表我们用来逐渐优化预测的混合损失。 λ是控制辅助监督项权重的超参数。


我们的实验是在四个公共光场基准数据集上进行的:LFSD [17]、HFUT [49]、DUTLF-FS [51] 和 DUTLF-V2 [29]。 HFUT 和 LFSD 相对较小,分别仅包含 255 个和 100 个样本。 DUTLF-V2是最大的数据集,包含4204个样本,分为2957个和1247个分别用于训练和测试。 DUTLF-FS包含1462个样本,分别分为1000个训练样本和462个测试样本。 每个样本都包含一个全焦点图像、几个焦点切片以及相应的地面实况显着性图。

为了公平比较,我们遵循大多数以前的方法 [31, 21] 使用 DUTLF-FS 和 HFUT 的训练集来训练我们的 FPT,以便与使用焦点堆栈输入训练的其他光场方法进行比较。 我们按照 [37, 14] 使用 DUTLF-V2 的训练集来训练我们的模型,以便与使用多视图输入训练的其他最先进的 LFSOD 方法进行比较。 我们将每个图像的大小调整为 256 × 256,以便于在训练和测试中实现,并且我们还通过随机翻转、裁剪和旋转来增强训练数据。
我们使用预训练的 PVT-B2 [39] 模型作为我们的主干,因为它具有与 ResNet50 [12] 相似的计算复杂性。 我们共享全焦点和焦点堆栈模式之间的主干权重,以减少不必要的参数。 整个网络使用 Adam [15] 作为优化算法进行端到端训练,并将初始学习率设置为 1e-4。 小批量大小设置为 2,我们的网络训练了 80 个时期。 学习率在第 40 和 70 epoch 分别乘以 0.1。 所提出的方法是使用Pytorch工具箱[26]实现的,所有实验都在四个RTX 1080Ti GPU上进行。 我们的代码将可用。 定量比较:为了进行全面比较,我们将我们的方法与 20 个最先进的模型进行比较,包括 7 个 LFSOD 方法:DLGLRG [21]、RENet [31]、LFNet [50]、MAC [47]、MoLF [51]和DLSD[30]; 6种RGB-D SOD方法:DCF [13]、CIR-Net [4]、VST-rgbd [20]、BBS-Net [10]、SSF [52]和S2MA [19]; 和 7RGB 方法:VST-rgb [20]、PFSNet [23]、ITSD [58]、LDF [44]、MINet [24]、F3 Net [43] 和 EGNet [56]。 为了保证公平比较,我们使用他们提供的显着性预测图或预训练的权重来生成比较数据,并利用[20]提供的相同评估代码。 如表 1 所示,很明显,所提出的方法在 DUTLF-FS 和 HFUT 数据集上实现了比当前最先进的方法更优越的性能。 同时,所提出的方法可以在超过Sα的三个指标上超越其他方法。 值得一提的是,与使用大量数据集训练的 RGBD 和 RGB SOD 方法相比,该方法在小三倍的训练集(1100 vs. 2985 vs. 10553)下取得了显着的优势。 这表明我们的方法可以有效地探索光场数据中传达的信息。 我们还在 DUTLF-V2 上重新训练我们的方法,并与其他三种使用多视图图像作为输入的 LFSOD 方法进行比较,包括 OBGNet [14]、LFBCNet [37] 和 ESCNet [53]。 如表 2 所示,我们的方法可以在 DUTLF-V2 的所有四个指标上大幅实现最佳性能。 这证明了使用焦点堆栈作为输入的优越性。 定性比较:为了更直观地观察,图4中可视化了所提出的方法和其他排名靠前的方法生成的一些代表性结果。可以看出,我们提出的方法的结果与地面事实更加一致。 当面对这些具有挑战性的场景时,包括多个对象(第 4、5 行)和复杂场景(第 6,7 行),大多数 RGB 和 RGB-D SOD 方法无法准确检测显着对象。 相比之下,所提出的方法可以成功生成准确且稳健的显着图。 与这些基于 CNN 的 LFSOD 方法相比,所提出的方法获得了更一致的预测结果和更精细的细节。


\BiSection{实验结果与分析}{TODO}

\BiSubsection{实验设置}{TODO}
\BiSubsection{消融实验}{TODO}
\BiSubsection{对比实验}{TODO}

不同模型组件的有效性。 我们首先验证表 3 中不同模型组件的有效性。我们首先构建一个基线模型。 具体来说,它使用共享权重编码器来提取全焦点和焦点堆栈特征,然后直接连接它们并预测显着性图。 接下来,我们逐渐将我们提出的 TCM、FPE 和 CCM 采用到我们的 FPT 模型中。 这三个模型分别表示为“+TCM”、“++FPE”和“+++CCM”。 如表3所示,这三个模型可以逐渐提高LFSOD性能,最终大幅优于基线模型。 此外,为了证明我们TCM中TI和TS子模块的有效性,我们逐渐在“+TCM”模型中采用TI和TS,并将它们表示为“+TI”和“++TS”。 与“基线”相比。 “+TI”和“++TS”在 DUTLF-FS 上分别使 MAE 指标降低 29.2\% 和 43.8\%。 我们还在图5中提供了每个消融研究相应的可视化结果。可以发现,预测掩模与地面实况图变得更加一致。 显着对象的完整内容和精确的显着图边界证明我们的 FPT 模型可以过滤掉背景干扰并更多地关注显着对象。 焦点相关标记的数量。 在表 4 中,我们用焦点相关标记的数量从 8 个增加到 32 个来测试我们的方法。与较少的焦点相关标记 (M = 8, 16) 相比,更多的焦点相关标记 (M = 32) 可以获得更好的结果 。 除非另有说明,我们的实验均使用 32 个与焦点相关的标记进行。 焦点感知增强中的 K 值。 我们设置了多个比较实验来选择表 5 中的最佳 K 数。当我们将 K 从 1 增加到 5 时,性能逐渐提高。 当 K > 5 时,我们观察到模型性能饱和并变得次优。 我们认为较大的 K 值会引入更多不显着的背景效应。 考虑到性能和计算成本之间的权衡,我们选择配置 K = 5 作为最终配置。

在本文中,我们提出了一种用于精确光场显着物体检测的焦点感知变压器(FPT)。 我们引入与焦点相关的令牌来收集图像特定的特征,并提出令牌通信模块(TCM)来传播信息并促进空间上下文建模。 为了增强焦点堆栈的空间特征表示,我们提出了一种焦点感知增强(FPE)策略来帮助抑制噪声背景信息。 实验结果表明,我们提出的方法可以在大多数 LFSOD 数据集上实现最先进的性能。

\BiSection{本章小结}{TODO}
%\BiChapter{基于视角增强的光场显著性目标检测}{TODO}
\label{chap:part4}
%
%
%
%
第三章从焦点感知的角度出发,设计了一种切片级探索多视角场景聚焦信息的显著性目标检测算法。
该方法注重对多视角三维场景的感知,以及不同视角对显著性预测的贡献程度,
实现了光场信息的深度挖掘以及多视角信息的高效探索。
%
%
本章从视角强化的角度出发,基于注意力机制引入视角增强模块,并通过前背景的补偿模块优化网络的训练,
实现了光场信息的充分挖掘。
%
%
%
%
\BiSection{研究动机}{TODO}
%
%
%
%
光场技术可以完整地记录场景的几何信息。在其中,焦点堆栈数据是光场数据的关键表达形式之一。现有研究表明,光场数据在显著性目标检测方面具有优势\cite{piao2019saliency,zhang2020light,wang2019deep,zhang2019memory,zhang2020lfnet,piao2021panet}。
随着基于Transformer架构的模型在各种视觉任务上取得超越性的性,
基于Transformer架构的光场显著性检测网络也逐渐出现\cite{wang2023tenet,liu2023lftransnet}。
然而,直接应用Transformer架构到光场显著性检测任务中,
并不能充分发挥Transformer架构对于长距离建模的能力,
不能得到理想的光场显著性目标检测效果。
合适的网络结构有待探索,加强模型对光场中隐含空间场景的感知,
从而获得鲁棒性的光场显著性分割预测。
% 使得模型对光场显著性检测有一个鲁棒性的结果。
%
%
%
%
\par
%
%
\begin{figure}[!h]
	\centering
	\includegraphics[width=0.95\linewidth]{figures/chapter4/task2_ins.drawio}
	\bicaption{
		光场显著性检测网络中不同的Transformer注意力机制
%		光场模型范例
	}{
		Different Transformer attention mechanisms in light field saliency detection network
%		TODO
	}  
	\label{cpt4_fig1:task2_ins}
\end{figure}
%
%
%
%
王等人~\cite{wang2023tenet}通过拼接焦点堆栈和全聚焦图特征,
一并送入Transformer编码器来建立光场整体结构的感知模型。
如图~\ref{cpt4_fig1:task2_ins}~(a)~所示,
但是,这种方式弱化了全聚焦图片特征和散焦图片特征之间的模态差异,两个模态之间的融合依然依赖后续的融合模块。
刘等人~\cite{liu2023lftransnet}只在焦点堆栈支路使用了Transformer结构。该方法聚合多尺度的焦点堆栈特征构造注意力矩阵,用一个可学习的权重来作为查询矩阵。通过注意力运算来汇总不同切片对显著性检测的贡献。
网络结构如图~\ref{cpt4_fig1:task2_ins}~(b)~所示。
%
%
这种方法没有充分考虑两个模态之间的差异,仅在单焦点堆栈模态起到特征强化效果,只能考虑有限的特征表达。
% 
% 
% 
% 
\par
%
%
针对上述问题,本章提出一种基于视角增强的光场显著性目标检测方法。
此方法致力于从焦点堆栈和全聚焦图的协同感知入手,结合Transformer强大的注意力机制,提出了视角增强的注意力方法,网络结构如图~\ref{cpt4_fig1:task2_ins}~(c)~所示。
我们在一个简单的骨干网络之上,添加了视角增强编码器,
首先两个支路先进行交叉注意力计算,进行一个初步的互感知;
再用从焦点堆栈特征提取共有的显著性视角表达,注入到以全聚焦图为主的交叉注意力模块内,促进对全聚焦图的注意力权重向共有显著性表达迁移。
之后将全聚焦图的特征注入到焦点堆栈特征来增强每一张焦点堆栈图像的视角表达。
通过使用掩码注意力,注意力被限制在以全聚焦预测片段为中心的显著性特征上。
与标准的Transformer解码器中使用的交叉注意力(关注图像中的所有位置)相比,
被屏蔽的注意力能够达到更快的收敛效果和性能提高。
其次,为了进一并提高网络的表达能力,探索了使用基于图片像素的对比学习策略,
来引导网络学习差异化的显著性语义和背景语义信息。
%
%
\par
% 
% 
% 
% 
为验证本方法的有效性,本章在 DUTLF, LFSD 以及 HFUT-LFSD 数据集上进行了实验,
同样获得了超越其他光场显著性分割网络的性能,为光场数据的应用奠定了基础。
% 
% 
% 为了进一步验证本方法的有效性,本章将提出的数据增强方法应用在当前最
% 好的光场显著性检测方法上,实验结果表明,本章的数据增强方法能够提升其它方法的
% 检测性能。
% 
% 
% 在视角增强编码器中,使用了掩码注意力,将
% ~
% 随着
% 然而,采集光场数据需要多组摄像头或相机矩阵,造成了高昂的成本。为了实现理想的光场显著性目标检测能力,除了设计合适的网络模型外,还需要大量高质量的光场数据。由于光场数据获取成本高,目前的方法通常采用数据增强手段来增加训练数据。传统的数据增强方法包括旋转、平移、裁剪和缩放等操作,但这些方式的变化有限,效果也不够显著。对于不同类型的光场数据,需要不同的增强方法,选择不当的方法可能会引入噪声,导致训练不稳定,降低模型性能。因此,如何稳定生成高质量的光场数据以实现数据增强的目标是当前需要解决的问题。
% 针对上述问题。  
% % task2_ins.drawio
% 
% 
% 
% 
\BiSection{方法介绍}{TODO}
% 
% 
%
%
%
%
%
\begin{figure}[!ht]
	\centering
	\includegraphics[width=0.95\linewidth]{figures/chapter4/chpt4_overview}
	\bicaption{基于视角增强的光场显著性检测网络}{
		Light field saliency detection network based on perspective enhancement
	}  
	\label{cpt4_fig1:chpt4_overview}
\end{figure}
%
%
%
%
%
本节将对本章提出的光场显著性目标检测网络模型进行详细介绍。
整体网络架构如图~\ref{cpt4_fig1:chpt4_overview}
~所示。
本章采用了在显著性目标检测任务中经常被使用的PVT-v2作为骨干网络。
具体来说,给定一个尺寸为$W \times H$的RGB图像$I_{0}$和
焦点堆栈$\left \{  I_{1},I_{2},I_{3},\cdots,I_{12} \right \} $,
分别输入编码器中以提取分辨率为$\frac{W}{2 \times 2^{l}} \times \frac{H}{2  \times 2^{l}} $ 
的特征$\left \{ F_{i}^{l},~l=1,2,3,4 \right \}$,
其中,$F_{i}^{l}$表示图像$I_{i}$在编码器第$l$层的特征,当$i=0$时,
$F_{i}^{l}$表示全聚焦图的特征,当$i=1,2,\cdots,12$时,$F_{i}^{l}$表示焦点堆栈图像的特征。
两个支路的编码器共享权重。由于使用注意力计算是图像尺寸的平方级复杂度,
我们只在特征尺度$\left \{ F_{i}^{l},~ l = 2, 3, 4\right \}$进行视角增强注意力计算。
增强后的全聚焦特征$ E_{0}^{l} $ 和焦点堆栈特征 $\left \{ E_{i}^{l},~i=1,~2,~ \cdots,~12 \right \}$
被送入特征融合模块,进行特征整合。
为了增强模型对显著性物体的辨识能力,我们采用了一个辅助显著性目标检测头,
用来生成对输入图片中不同像素位置的特征表示。并通过放大显著性前景区域与背景区域的差异,来辅助训练。
之后显著性分割头利用整合后的特征和骨干网络输出的第一层多尺度
特征$\left \{ F_{i}^{1},~i=0,~1,~ \cdots, ~12 \right \}$进行融合,并生成最终的显著性预测。
%
%
%
%
%参考CPD的建议,低阶特征由于太简单而无法做出有效且可靠的预测,
%本章只在高阶特征(即$F_{i}^{l},~j=2,3,4,5$)上执行解码操作。
%
%\textcolor{red}{TODO}
%
本章工作提出的视角增强模块和跨图的像素对比学习策略将
分别在~\ref{chap:part4_view_enh}~以及~\ref{chap:part4_cons}~节进行详细介绍。
%
%
%
%
\BiSubsection{视角增强模块}{TODO}
%
%
%
%
\label{chap:part4_view_enh}
% 
% 
% 
%
%
\begin{figure}[!ht]
	\centering
	\includegraphics[width=0.95\linewidth]{figures/chapter4/view_enhance}
	\bicaption{
		视角增强注意力模块
	}
	{Perspective Enhanced Attention Module}  
	\label{cpt4_fig1:view_enhance}
\end{figure}
%
%
%
%
\par 
%
%
%
%
视角增强模块综合考虑全聚焦图片的整体清晰视角和每一张焦点堆栈图片的局部清晰视角,
利用协同注意力的方式寻找全聚焦特征和焦点堆栈特征中对显著性目标检测最有效的语义信息。
视角增强注意最理想的情况是协同注意力权重往有效区域分布。
本节工作的目的是生成显著性视角表示,并对两个模态的互注意力权重进行加权以突出
显著性区域表示。


增强两个模态的特征表示,
一个直接的方式是使用全聚焦图和焦点堆栈的特征输入交叉注意力模块来强化特征,
但是,光场数据有限,模型容易陷入局部优化区。
并且,散焦区域特征和聚焦区域特征都是拍摄的同一场景,在特征层面相差不大,
互注意力模型在训练过程中往往迭代缓慢,且易生成较差的权重矩阵~\cite{piao2021panet}。
本章提出视角增强模块,通过在进行全聚焦特征和焦点堆栈特征
的交叉注意力权重时,使注意力限制在共有显著区域内部,
从而在特征整合中突出有效的聚焦显著性区域。


具体来说,
从共享权重的骨干网络中获取到四层特征$\left \{ F_{i}^{l},~l=1,2,3,4 \right \}$后,
考虑到Transformer块对于图像尺寸是一个平方级的计算复杂度,
使用越低级的语义信息,会带来更高的计算复杂度。
我们采取一种折中的办法,
只使用原始图像分辨率为$1/32,~1/16$和$1/8$的像素编码器产生的特征金字塔。
对于每一个分辨率,在输入到Transformer解码器之前加入正弦位置嵌入
$ e_{pos}\in \mathbb{R}^{H_{l}W_{l}\times C} $。
使用的Transformer解码器层,使用从次低分辨率到最高分辨率的三层特征,
如图~\ref{cpt4_fig1:chpt4_overview}~所示。
网络总共重复这个3层的Transformer解码器$L$次。因此,网络最终的Transformer解码器层有$3L$层。
更具体的说,前三层介绍分辨率$H_{1}=H/32,H_{2}=H/16,H_{3}=H/8$
和$W_{1}=W/32,W_{2}=W/16,W_{3}=W/8$的特征图,其中$H$和$W$是原始图像分辨率。
之后的所有层都以循环的方式重复此模式。



对全聚焦特征$ F_{0}^{4} $和
焦点堆栈特征$ \left \{ F_{i}^{4}, i=1,2, \cdots, 12 \right \}$
进行自注意力计算以增强其模态内的特征表示。
自注意力最早来源于Transformer架构~\cite{vaswani2017attention},
给定一个查询元素(例如,输出句子中的目标单词)和一组关
键元素(例如,输入句子中的源单词),多头注意力模块根据衡量查询密钥对兼容性
的注意力自适应地聚合关键内容。
为了使模型能够关注来自不同表示子空间和不同位置的内容,不同注意力头的输出与可学习的权重进行线性聚合。
%
%
%
%为了让模型关注来自不同子空间和不同位置的内
%容,不同注意力头的输出与科学系的权重进行线性聚合。
%
%
%
%Transformer 中的多头注意力。
%Transformer最早用于机器翻译的,是一种基于注意力机制的网络架构。
%给定一个查询元素(例如,输出句子中的目标单词)和
%一组关键元素(例如,输入句子中的源单词),
%多头注意力模块根据衡量查询密钥对兼容性的注意力自适应地聚合关键内容。
%为了让模型关注来自不同子空间和不同位置的内容,不同注意力头的输出与科学系的权重进行线性聚合。
%
%
%
令$q\in \Omega_{q}$索引具有表示特征$z_{q} \in \mathbb{R}^{C}$的查询元素,
$k\in \Omega_{k}$索引具有表示特征$x_{k} \in \mathbb{R}^{C}$的关键元素,
其中$C$是特征维度,$\Omega_{q}$和$\Omega_{k}$分别指定查询元素和关键元素的集合。
然后计算多头注意力特征的公式如下:
% 
% 
% 
% 
\begin{equation}
	MultiHeadAttn(z_{q},~x)=\sum_{m=1}^{M}
	W_{m}\left [ ~~\sum_{k\in \Omega_{k}}^{}A_{mqk} ~\cdot~ W_{m}^{'} x_{k}  ~~\right ]  
\end{equation}
% 
% 
% 
% 
其中$m$是注意力头的索引,
$ W_{k}^{{}' } \in \mathbb{R}^{C_{v} \times C} $
和$W_{m} \in \mathbb{R}^{C\times C_{v} }$
是可学习的权重(默认情况下$C_{v}=C/M$)。
注意力权重 
$A_{mqk} \propto exp \left \{ 
\frac{
	z_{q}^{T} U_{m}^{T} V_{m} x_{k}
}{ \sqrt{C_{v}}
}  \right \} $
% 
% 
% 
归一化为$ {\textstyle \sum_{k\in\Omega_{k}}^{}} A_{mqk} = 1$,
其中$U_{m},~V_{m} \in \mathbb{R}^{C_{v}\times C}$也是可学习权重。
为了消除不同空间位置的歧义,表示特征$z_{q}$和$x_{k}$通常是元素内容和位置嵌入的串联求和。
%
%
%
%
%
%
\par
%
%
%
%
但是,只进行模态内部的自注意计算,无法感知到隐含在焦点堆栈内部的三维场景信息。
还需要焦点堆栈和全聚焦图像进行互相的特征增强,其他的特征融合增强的范式可以见图~\ref{cpt4_fig1:task2_ins}。
最近的研究~\cite{gao2021fast,sun2021rethinking}~表明,
基于Transformer的模型收敛缓慢是由于交叉注意力层中的全局上下文,
因为交叉注意力需要许多训练周期才能学会关注局部对象区域~\cite{sun2021rethinking}。
局部特征足以更新查询特征,并且可以通过自注意力收集上下文信息~\cite{cheng2022masked}。
标准的交叉注意力(带残差路径)计算公式如下:
\begin{equation}
	X_{l}=Softmax(Q_{l}K_{l}^{T})V_{l} + X_{l-1}
\end{equation}
其中,$l$是层索引,$X_{l} \in \mathbb{R}^{N\times C}$
是指在$l^{th}$层和
$ Q_{l} = f_{Q} \left ( X_{l-1} \right ) \in \mathbb{R}^{N \times C} $
的维度查询特征。$X_{0}$表示Transformer解码器的输入查询特征。
$K_{l},V_{l} \in \mathbb{R}^{H_{l}W_{l} \times C}$
分别表示经过$f_{K}(\cdot) $和$ f_{V}( \cdot )$变换后的
输入图像特征,
$H_{l}$和$W_{l}$是图像特征的空间分辨率。
$f_{Q} (\cdot),~f_{K}(\cdot) $和$ f_{V}(\cdot) $是线性变换。
$ Softmax(\cdot) $ 函数将输入的权重矩阵映射为 $(0,~1)$之间的分布,
并且使得输出值的累计和为1,从而满足概率性质,即对概率最大的元素加强注意。
%
%
%
%
%
\par
%
%
%
%
这些问题在光场中会更加严重。
在光场中应用交叉注意力机制,
会使得注意力权重往散焦图片中清晰的部分偏移,
因为散焦图片中既有前景清晰的部分,也有背景清晰的部分,
网络很容易被背景清晰的部分误导,从而产生错误的显著性预测结果。
且网络需要更多的迭代训练才能对是否是前景显著物体有一个很好的辨识度。
%
%
%
%
%
\par
%
%
%
%
针对上述注意力弥散的问题,
本节提出掩码交叉注意力,可以看做是交叉注意力机制的一种变体,
仅关注每个查询的预测掩码的前景区域内。
% 
% 
% 
% 
掩码注意力通过以下方式调节注意力矩阵:
\begin{equation}
	X_{l}=softmax(M_{l-1} ~+~Q_{l}K_{l}^{T})V_{l} + X_{l-1}
\end{equation}
% 
% 
此外,在特征位置$(x,~y)$的注意力掩码$M_{l-1}$是从另一个模态的特征提取的共有表示。
%
%
%
%
%
%
\par 
%
%
%
%
根据交叉注意力的方式不同,注意力掩码有两个生成方式,
对于全聚焦图特征$F_{0}^{l}$与焦点堆栈特征的交叉注意力,
使用来自焦点堆栈的共有表示来强化注意力权重。
对于给定焦点堆栈特征$\left \{ F_{i}^{l}, i=1,2,\cdots, 12 \right \}$,
进行如下变换:
%
%
%
%
\begin{equation}
\begin{aligned}
	\hat{F} &= Concat(\left [  F_{1}^{l}, F_{2}^{l}, \cdots, F_{12}^{l} \right ] ) \\
	M_{a}^{l} &= \bar{f}
	\left (  \sigma \left ( f^{1 \times 1}\left [  
	AvgPool\left ( \hat{F} \right ) ,~ MaxPool\left ( \hat{F} \right ) 
	\right ]  \right )  \right )
\end{aligned}
\end{equation}
%
%
%
其中$Concat(\cdot)$表示在通道维度进行拼接操作,
$\sigma(\cdot)$表示RELU激活函数,
$f^{1\times 1}$表示$1\times 1$通道卷积操作,
$AvgPool(\cdot)$
和$MaxPool(\cdot)$
分别表示通道维度上的平均池化和最大池化操作。
$\bar{f}$表示将得到的掩码进行阈值化处理,公式如下:
%
%
%
%
%
%
%
\begin{equation}
M^{l}\left ( F\left ( x, ~y \right )  \right ) =
\begin{cases}
	0  & \text{ if } ~~ F(x,~y)>  \alpha \\
	-\infty & \text{ otherwise } 
\end{cases}
\end{equation}
%%
%
%
%
其中$\alpha$是超参数,是对特征进行二值化的阈值,一般设为0.5。
%
%
%
%
%
%
\par
%
%
%
%
%
而对于焦点堆栈支路的掩码交叉注意力计算,
掩码使用来自全聚焦特征支路的特征表示来生成共有表示,其公式如下:
%
%
%
%
\begin{equation}
	M_{fs}^{l} = \bar{f}   \left ( \sigma \left ( 
	f^{1\times 1}
	\left ( 
%	\hat{F}
	F_{0}^{l}
	\right ) 
	\right )   \right ) 
\end{equation}
%
%
%
%
其注意力计算的的 $Q_{l}$ 是  $ \hat{F}$,
$K_{l}$ 
和
$V_{l}$是来自全聚焦支路的特征 $F_{0}^{l}$。
%
%
%
%
%
\par
%
%
%
%
语义分割任务中,上下文特征已经被证明对图像分割很重要
~\cite{chen2017deeplab,chen2017rethinking,zhao2017pyramid},
高分辨率的语义特征对提高模型性能具有显著帮助,特别是小物体的分割。
我们在特征融合阶段对于骨干网络中获取的最低尺度特征$\left \{ F_{i}^{0}, i = 0, 1, \cdots, 12 \right \}$加以考虑。
特征融合模块获取
视角增强模块的输出$\left \{ E_{i}^{l}, i = 0, 1, \cdots, 12 \right \}$
和最高分辨率的语义特征。具体的融合过程如下:
%
%
%
%
\begin{equation}
	\begin{aligned}
		E^{l+1} &= 
		Concat\left ( \left [ E_{0}^{l+1},E_{1}^{l+1},\cdots, E_{12}^{l+1} \right ]  \right )  
		\\  
		F^{l} &= 
		Concat\left ( \left [ F_{0}^{l},F_{1}^{l},\cdots, F_{12}^{l} \right ]  \right )  
		\\  
		{O^{l}}  &= Conv\left (  Upsample\left ( f^{3 \times 3} \left ( E^{l+1} \right )  \right ) + F_{0}^{l} \right )  \\
	\end{aligned}
\end{equation}
%
%
%
%
其中,$Concat(\cdot)$是通道拼接操作,
$f^{3 \times 3}$ 是 $3 \times 3 $ 卷积操作,
$ Upsample(\cdot) $ 是2倍上采样过程,
$ Conv $ 是一个组合操作,包含 $3\times 3$卷积、批归一化和RELU激活函数。
特征融合模块从高层语义特征汇总到低层语义信息,
层级表示$i$的取值依次为$\left \{ 3,2,1 \right \}$。
%
%
%
%
%
\par 
%
%
最后将获取到的融合模块的输出特征$\left \{ O^{l}, l = 1,2,3 \right \}$
送到显著性分割头进行最终的分割预测 $P$,
同时,为了增强网络的表示,我们采用多尺度监督,
分别在图像尺寸为1/16、1/8和1/4的原图像尺寸上预测显著性图
$\left \{ P^{l}, l = 1,2,3 \right \}$。
具体来说,采用如下公式:
%
%
%
%
\begin{equation}
	P^{l} = \sigma \left ( Conv\left ( MaxPool \left (  
	O^{l} \right )  \right )  \right ) 
	\times 
	O^{l} + O^{l},~ l = 1,2,3
\end{equation}
%
%
%
%
其中,$MaxPool(\cdot)$是通道维度的最大池化操作,
$ Conv $ 表示 $ 3\times 3 $卷积操作,
$ \sigma $ 代表的是 RELU 激活函数。
获取到的三个显著性预测图分别以真值图进行监督训练,
显著性分割损失
$ \mathcal{L}_{sal} $
定义为:
%
%
%
%
\begin{equation}
	\mathcal{L}_{sal} = \sum_{l = 1}^{3}  
	\mathcal{L}_{bce} \left ( P^{l}, DownSample_{j}\left (  GT \right ) 
	\right ), ~ j = 2\times 2^l
	\label{chpt4:eq:loss_sal}
\end{equation}
%
%
%
%
其中$\mathcal{L}_{bce}$ 表示二值交叉熵损失函数,
$DownSample_{j}(\cdot)$ 表示将输入下采样$j$ 倍。
%
%
%
%
%
\BiSubsection{感知对比学习策略}{TODO}
\label{chap:part4_cons}
% 
% 
% 
% 
显著性目标检测网络一般使用$sigmoid$ 作为输出结果的映射头,
这种通过阈值化分割前景和背景的方式,
虽然能够通过交叉熵损失优化前景和背景中每个像素点与真值预测图之间的距离,
但忽略了像素之间的关系~\cite{zhao2019region},
在一些置信度低的区域,比如前背景相似的像素区域,
网络容易产生错误的预测,或者在学习不到位时,会预测出虚影。
%
%
%
%
虽然显著性分割是一个像素级的分类任务,但是每张图像的前景区域内部和背景区域内部也是有差异所在。

在本章工作中,我们构造了一种逐像素对比学习的方法,
通过规范嵌入空间并探索训练数据的全局结构来解决上述问题。
使得网络能够在做出像素级显著性分割预测时,也能够考虑同类像素之间的关系,
学习显著性区域内部的一致性表达,
和背景区域内部的一致性表达,我们在光场显著性检测领域引入了
感知对比学习。



对比学习常用于无监督中视觉表示学习。
无监督对比学习旨在学习CNN编码器$f_{CNN}$将每个训练图像转换为特征向量表示$v=f_{CNN}(I) \in \mathbb{R}^{D}$,
使得$v$能够描述图片$I$。
为了实现这一目标,
对比学习通过区分正样本(一个增强版本的图像$I$)
和多个负样本(从训练集中随机挑选的图像,但是不包括$I$)来进行训练。
在对比学习中,会集成使用InfoNCE损失函数,其公式如下:
\begin{equation}
	\mathcal{L} _{NCE}^{I}=-log \frac{exp(v \cdot v^{+ }/\tau )}
{exp(v \cdot v^{+}/\tau )+ \sum_{v^{-}\in N_{I}} exp(v \cdot v^{-}/\tau )} 
\label{chpt4_equ_nce}
\end{equation}
其中$v^{+}$是正样本图像$I$的嵌入,$N_{I}$是包含负样本的嵌入,$\cdot$表示内(点)积,
$\tau >0$是温度超参数。损失函数在计算前,还需要对所有嵌入进行$\ell_{2}$归一化。
\par
% 
% 
% 
%% 
%另一个需要解释的概念是知识库。
%一些最近的研究表明,大量的负样本(即$N_{I}$)在无监督对比学习中至关重要
%\cite{wu2018unsupervised,chen2020improved,he2020momentum}。
%但是负样本的数量受到小批量(mini-batch)大小的限制,
%最近的对比学习方法利用大型外部存储器来储存更多的负样本。
%具体来说,一些方法\cite{wu2018unsupervised}直接将所有训练样本的嵌入表示存储在内存中,
%但是很容易受到异步更新的影响。
%其他一些人选择用一个队列保存最后几个批次的的嵌入\cite{wang2020cross,chen2020improved,he2020momentum}。
%在\cite{chen2020improved,he2020momentum}中,
%存储的嵌入表示还可以通过编码器网络$f_{CNN}$的动量更新而进行动态更新。
%\par
% 
% 
% 
% 

在显著性分割的背景下,图像$I$的每个像素$i$需要被分类为显著性的前景类或者背景类。
当前的方法通常将此任务视为逐像素分类问题。具体来说,
令$f_{FCN}$作为图像特征编码器(例如ResNet\cite{he2016deep}),它为图像$I$生成密集特征预测
$I\in \mathbb{R}^{ H \times W \times D}$,
从中可以导出每个像素的嵌入$i \in  \mathbb{R}^{D}$。



我们首先扩展了公式~\ref{chpt4_equ_nce},
去适配我们有监督的密集图像预测任务。
总的来说,我们的对比损失函数计算的是数据样本中每张图像的像素。
对于具有真实语义标签的像素$i$,正样本是也属于该类的其他像素,
而负样本是属于其他类的像素。
我们的有监督形式的逐像素对比损失定义为:
\begin{equation}
	\mathcal{L} _{NCE}^{I}= 
	\frac{1}{|P_{i}|}
	\sum_{i^{+}\in P_{i}}^{}  
	-log \frac
	{exp(i \cdot i^{+ }/\tau )}
	{exp(i \cdot i^{+}/\tau )+ \sum_{i^{-}\in N_{I}} exp(i \cdot i^{-}/\tau )} 
	\label{chpt4:eq:con_loss}
\end{equation}
% 
% 
% 
% 
% <<<<<<< HEAD
% 其中$m$索引注意力头,$k$索引采样键,
% $K$是采样键的总数($K \ll HW$),
% $\bigtriangleup p_{mqk} $和$A_{mqk}$分别表示
% 第$m^{th}$个注意力头中第$k^{th}$个采样点的采样偏移量和注意力权重。
% 标量注意力权重$A_{mqk}$位于$\left [ 0,~1 \right ] $范围内,
% 通过$ {\textstyle \sum_{k=1}^{K}} A_{mqk}=1$进行归一化。
% $\bigtriangleup p_{mqk} \in \mathbb{R}^{2}$
% 是范围不受约束的二维实数。
% 由于$p_{q} + \bigtriangleup p_{mqk}$是分数,如等人所述,
% $x\left ( p_{q} + \bigtriangleup p_{mqk} \right ) $在计算时应用双线性插值。
% $\bigtriangleup p_{mqk}   $和$A_{mqk}$都是通过查询特征$z_{q}$上的线性投影获得的。
% 在具体实现中,查询特征$z_{q}$被馈送到$3MK$通道的线性投影算子,
% 其中前$2MK$通道对采样偏移量$\bigtriangleup p_{mqk}$进行编码,
% 其余$MK$通道通过被馈送到$softmax$算子以获得注意力权重$A_{mqk}$。






% 可变形注意力模块旨在将卷积特征图作为关键元素进行处理。

% 设 为查询元素的数量,当较小时,可变形注意力模块的复杂度为。
% 当应用于DETR编码器时,其中,复杂度变为,
% 起复杂度与空间大小呈线性关系。
% 当它作为DETR解码器中的交叉注意力模块应用时,
% 其中,是对象查询的数量,复杂度变为,
% 这与空间大小无关。


% 多尺度可变形注意力模型。
% 大多数现代目标检测框架都受益于多尺度特征图。
% 我们提出的可变形注意力模块可以自然地扩展到多尺度特征图。
% 令为输入多尺度特征图,
% 其中。
% 令为每个查询元素的参考点的归一化坐标,
% 然后应用多尺度可变形注意力采样点。
% 分别表示第几个特征层和第几个注意力头中第几个采样点的采样偏移和注意力权重。
% 标量注意力权重通过进行归一化。

% 这里,为了尺度公式的清晰性,我们使用归一化坐标,
% 其中归一化坐标分别表示图像的左上角和右下角。
% 方程中的函数将归一化坐标重新放缩到第几层的输入特征图。


% 多尺度可变形注意力与之前的单尺度版本非常相似,只是它从多尺度特征图中采样点,
% 而不是从单尺度特征图中采样k个点。
% 当且固定为单位矩阵时,所提出的注意力模块将退化为可变形卷积。
% 可变形卷积是针对单尺度输入而设计的,每个注意力头仅关注一个采样点。
% 然而,我们的多尺度可变形注意力会关注来自多尺度输入的多个采样点。
% 所提出的(多尺度)可变形注意模块也可以被视为
% Transformer 注意的有效变体,其中通过可变形采样位置引入预过滤机制。
% 当采样点遍历所有可能得位置时,所提出的注意力模块相当于Transformer注意力。


% 可变形Transformer 编码器。

% 我们将网络中处理特征图的Transformer注意模块替换为所提出的多尺度可变形注意模块。
% 编码器的输入和输出都是具有相同分辨率的多尺度特征图。

% 在编码器中,我们从骨干网络中提取多尺度特征图,通过卷积,
% 其中的分辨率比输入图像低。
% 最低分辨率特征图是通过最后阶段的步长卷积获得的。

% 所有多尺度特征图均为个通道,
% 像FPN网络结构中,没有使用自上而下的结构,因为我们提出的多尺度可变形注意力本身可以在多尺度特征图之间交换信息。

% =======
其中$P_{i}$和$N_{i}$分别表示像素$i$的正样本和负样本的像素嵌入集合。
并且,正负样本和锚点$i$不局限于统一图像。如公式~\ref{chpt4:eq:con_loss}~所示,
这种基于像素到像素对比度的损失设计的目的是通过
将同一类像素样本拉进
并
将不同类像素样本推开来学习其隐含的嵌入表示。





% 
% 
公式~\ref{chpt4:eq:loss_sal}~中的像素二元交叉熵损失
$\mathcal{L}_{sal} $
以及
在公式~\ref{chpt4:eq:con_loss}~中的对比损失
$\mathcal{L}_{NCE} $
是互补的。
前者能让显著性分割网络学习对前背景分类有意义的判别性像素特征,
而后者通过显示探索像素样本之间的全局语义关系,
有助于规范嵌入空间,
提高类内紧凑性和类间可分离性。
因此总体训练目标是:
\begin{equation}
	\mathcal{L}_{total} = \sum_{i}^{} \mathcal{L}_{sal}^{i} + \mathcal{L}_{NCE}^{i}
%	\mathcal{L}^{SEG}=\sum_{i}\left ( 
%	\mathcal{L}_{i}^{CE} + 
%	\lambda \mathcal{L}_{i}^{NCE}  
%	\right ) 
\end{equation}
% 
% 
% 
% 
其中$\lambda > 0 $是系数,$\mathcal{L}^{total}$
学习到的像素嵌入变得更加紧凑且分离良好。
这表明,通过利用二元交叉熵损失和成对比度量损失的优势,
显著性分割网络可以生成更具辨别力的特征,
从而产生更有希望的结果。


%
%
%\todo 





% 
% 
% 
% 
\BiSubsection{训练过程}{TODO}



本文中提到的
感知对比表示头,如图~\ref{cpt4_fig1:chpt4_overview}~中所示,
和对比损失 $\mathcal{L}_{NCE}$
只在训练过程中存在。
在测试阶段,使用显著性分割头预测的最后一个预测 $P^{1}$, 再通过双线性插值到原图大小后,
经过$sigmoid(\cdot)$激活后作为最终的显著性分割预测。




%\\
%\\
%\\
%\\
\BiSection{实验结果与分析}{TODO}

\BiSubsection{实验设置}{TODO}
(1)数据集

本章在 3 个公开的光场数据集:
DUT-LFSD~\cite{zhang2019memory}、
HFUT-LFSD~\cite{zhang2017saliency}以及 
LFSD~\cite{li2014saliency}上
进行了大量的实验。DUT-LFSD 包含 1462 张光场样本,HFUT-LFSD 包含 255 个样本,
LFSD 包含 100 个光场样本。每个样本包含一个 RGB(全聚焦)图像,焦点堆栈以及一
个像素级的显著性真值。本章采用了与 ERNet~\cite{piao2020exploit}~ 相同的训练集。具体来说,本章从
DUT-LFSD 中选择了 1000 个样本,并从 HFUT-LFSD 中选择了 100 个样本作为训练集。

(2)实现细节

本章的方法是在 Pytorch 框架上实现的,并使用单个 GTX2080Ti 的显卡进行训练。在训
练过程中,本章使用 AdamW 优化器,并将优化器中的动量以及权重衰减参数设置为 0.9
以及 0.00005。
优化器的学习率设置为 3e-5,并在前 5 万次迭代过程中逐渐衰减并最终
衰减为原学习率的 1/10。
小批量尺寸大小设定为 2,并将网络中批量归一化(BatchNorm, BN)修改为组
归一化 (GroupNorm,GN)。
最大迭代次数设置为 10 万次,所有光场数据大小均调整到 256×256后进行训练和测试。

(3)评价指标

为了公平的比较不同网络的性能,本章采用了 S-measure、E-measure、F-measure、绝对平均误差MAE作为评
估模型性能的指标。

\BiSubsection{消融实验}{TODO}

在本小节,对提出的各个模块进行消融实验,
分析每个组件或者策略对于显著性分割性能的影响。
定量的消融实验结果展示在表,
定性的消融实验结果展示在图。

(1)
验证视角增强模块的有效性




(2)验证像素对比监督的有效性





\BiSubsection{对比实验}{TODO}

本节将本章方法与其他数据增强方法进行了比较:

	%------------------------------ figure: comparison
\begin{figure*}
	\centering
	% \setlength{\abovecaptionskip}{-5mm}
	\includegraphics[width=\linewidth]{figures/chapter3/compare_3}
	%	\caption{
		%		Qualitative comparisons of state-of-the-art methods in some challenging scenes, including multiple objects and complex scenes.
		%	}
	\bicaption{
		在一些具有挑战性的场景(包括多物体和复杂场景)中对最先进的方法进行定性比较。
	}{
		Qualitative comparisons of state-of-the-art methods in some challenging scenes, including multiple objects and complex scenes.
	}
	\label{chpt4:fig:comparison_3}
	\vspace{-0.2cm}
\end{figure*}
% 
%

(1)定性比较


图~\ref{chpt4:fig:comparison_3}~展示来本章方法与目前最先进的显著性目标检测方法在
不同场景的显著性预测的可视化效果。
\todo 
在1-3行中,显著性目标在不同深度范围中表现突出;4-5行描述了多变的背景情况;6-7行展示了显著性目标与背景相近的挑战。我们的方法可以准确且完整地预测这些复杂场景中的显著性目标。举例来说,EAR、LFNet和MOD等光场显著性检测算法可以准确定位目标,但无法完整预测;CPFP和PCA等RGB-D方法在有高质量深度图的场景中表现良好,但容易受到低质量深度图的干扰;CPD和PoolNet等RGB方法由于缺乏空间信息,在复杂场景中难以准确预测。
%
%
%
%
%
这些结果显示出我们提出的视角增强模块的光场数据探索方法更有效地利用多焦点数据,准确定位显著区域,
并在特征整合中突出显著区域、抑制非显著区域。


\begin{table}
	%	\caption{Ablation analyses of each component on the DUTLF-FS dataset.
		%		The best results are marked in \textbf{boldface}.
		%	}
	\bicaption{
		DUTLF-FS 数据集上每个组件的消融分析。
	}{
		Ablation analyses of each component on the DUTLF-FS dataset.
	}
	\centering
	\label{chpt4:tab:abl_1}
	%	\resizebox{0.82\linewidth}{!}{
		\begin{tabular}{lcccccccc}
			\toprule  %添加表格头部粗线
			%%  \multicolumn{1}{c}{ \multirow{2}*{Methods} }
			
			\multicolumn{1}{c}{ \multirow{2}*{模型设置}}	& \multicolumn{4}{c}{DUTLF-FS} & \multicolumn{4}{c}{HFUT} \\ 
			
%			\cmidrule(r){2-9} 
			
			\cmidrule(r){2-5} \cmidrule(r){6-9} 
			
			& $E_{\phi}^{max}\uparrow$ & $S_{\alpha }\uparrow $ & $F_{\beta}^{max}\uparrow$ & MAE$\downarrow$ 
			& $E_{\phi}^{max}\uparrow$ & $S_{\alpha }\uparrow $ & $F_{\beta}^{max}\uparrow$ & MAE$\downarrow$
			\\
			
			\midrule
			
%			% 开始填写数据
%			\multicolumn{2}{l}{ Baseline }     & 0.947 & 0.894 & 0.901 & 0.048 \\ 
%			
%			\midrule
		
			
			+自注意力 
			& 0.964 & 0.926 & 0.935 & 0.031 
			& 0.881 & 0.833 & 0.797 & 0.065   \\
			
			+交叉注意力
			& 0.965 & 0.931 & 0.943 & 0.027 
			& 0.881 & 0.833 & 0.795 & 0.060   \\
			
			+自掩码注意力  
			& 0.965 & 0.935 & 0.944 & 0.025 
			& 0.894 & 0.849 & 0.821 & 0.058   \\
			
			+掩码注意力   
			& 0.971 & 0.94  & 0.949 & 0.022 
			& 0.902 & 0.852 & 0.823 & 0.052  \\
			
			\bottomrule
		\end{tabular}
		% }
\end{table}

\begin{table}
	%	\caption{Ablation analyses of each component on the DUTLF-FS dataset.
		%		The best results are marked in \textbf{boldface}.
		%	}
	\bicaption{
		DUTLF-FS 数据集上每个组件的消融分析。
	}{
		Ablation analyses of each component on the DUTLF-FS dataset.
	}
	\centering
	\label{chpt4:tab:abl_2}
	%	\resizebox{0.82\linewidth}{!}{
		\begin{tabular}{llcccc}
			\toprule  %添加表格头部粗线
			%%  \multicolumn{1}{c}{ \multirow{2}*{Methods} }
			
			\multicolumn{2}{c}{ \multirow{2}*{Settings}}	& \multicolumn{4}{c}{DUTLF-FS} \\ %& \multicolumn{3}{c}{HFUT} \\ 
			
			\cmidrule(r){3-6} 
			
			& & $E_{\phi}^{max}\uparrow$ & $S_{\alpha }\uparrow $ & $F_{\beta}^{max}\uparrow$ & MAE$\downarrow$ \\
			\midrule
			
			% 开始填写数据
			\multicolumn{2}{l}{ Baseline }     & 0.947 & 0.894 & 0.901 & 0.048 \\ 
			
			%		 			\multicolumn{2}{l}{+Token} 	 & 0.959 & 0.918 & 0.926 & 0.037 \\ 
			
			\midrule
			
			\multicolumn{1}{c}{ \multirow{2}*{+TCM}}	
			
			& +TI		& 0.961 & 0.923 & 0.932 & 0.034 \\ 
			& ++TS & 0.968 & 0.933 & 0.944 & 0.027 \\
			\midrule
			
			\multicolumn{2}{l}{++FPE} 		& \textbf{0.972} & 0.941 & 0.952 & 0.022 \\
			\multicolumn{2}{l}{+++CCM} 		& \textbf{0.972} & \textbf{0.942} & \textbf{0.953} & \textbf{0.021} \\ 
			
			
			\bottomrule
		\end{tabular}
		% }
\end{table}

\begin{table}
	%	\caption{Ablation analyses of each component on the DUTLF-FS dataset.
		%		The best results are marked in \textbf{boldface}.
		%	}
	\bicaption{
		DUTLF-FS 数据集上每个组件的消融分析。
	}{
		Ablation analyses of each component on the DUTLF-FS dataset.
	}
	\centering
	\label{chpt4:tab:abl_3}
	%	\resizebox{0.82\linewidth}{!}{
		\begin{tabular}{llcccc}
			\toprule  %添加表格头部粗线
			%%  \multicolumn{1}{c}{ \multirow{2}*{Methods} }
			
			\multicolumn{2}{c}{ \multirow{2}*{Settings}}	& \multicolumn{4}{c}{DUTLF-FS} \\ %& \multicolumn{3}{c}{HFUT} \\ 
			
			\cmidrule(r){3-6} 
			
			& & $E_{\phi}^{max}\uparrow$ & $S_{\alpha }\uparrow $ & $F_{\beta}^{max}\uparrow$ & MAE$\downarrow$ \\
			\midrule
			
			% 开始填写数据
			\multicolumn{2}{l}{ Baseline }     & 0.947 & 0.894 & 0.901 & 0.048 \\ 
			
			%		 			\multicolumn{2}{l}{+Token} 	 & 0.959 & 0.918 & 0.926 & 0.037 \\ 
			
			\midrule
			
			\multicolumn{1}{c}{ \multirow{2}*{+TCM}}	
			
			& +TI		& 0.961 & 0.923 & 0.932 & 0.034 \\ 
			& ++TS & 0.968 & 0.933 & 0.944 & 0.027 \\
			\midrule
			
			\multicolumn{2}{l}{++FPE} 		& \textbf{0.972} & 0.941 & 0.952 & 0.022 \\
			\multicolumn{2}{l}{+++CCM} 		& \textbf{0.972} & \textbf{0.942} & \textbf{0.953} & \textbf{0.021} \\ 
			
			
			\bottomrule
		\end{tabular}
		% }
\end{table}





(2)定量比较

%
%
\begin{table*}[!ht]
	%
	%---------------------------------------------------------------------> 大表 
	%
	%	\caption{Quantitative comparison of our proposed FPT with other 20 SOTA SOD methods on three benchmark datasets. 
		%		$ \uparrow \& \downarrow $ denote larger and smaller is better.
		%		%
		%		% denote the best and the second-best results,
		%		%
		%		The best three results are shown in 
		%		\textbf{boldface}, \textcolor{red}{red} and \textcolor{blue}{blue} fonts respectively. 
		%		% '-' indicates the code or outcome is not available.
		%	}
	\bicaption{
		在 3 个公开数据集上的定量比较
	}{Quantitative comparisons on three light field datasets}
	%	\renewcommand{\arraystretch}{1.5}
	
	\centering
	\label{chpt4:table:comp_with_sota_3_1}
	\resizebox{\textwidth}{!}{
		\begin{tabular}{rcccccccccccc}
			\toprule  %添加表格头部粗线
			
			% title
			%			\multirow{2}*{Type} & 
			\multicolumn{1}{c}{ \multirow{2}*{方法} } & 
			\multicolumn{4}{c}{DUTLF-FS \cite{zhang2019memory} } &
			\multicolumn{4}{c}{HFUT \cite{zhang2017saliency} } &
			\multicolumn{4}{c}{LFSD \cite{li2014saliency} } \\
			
			% next line
			\cmidrule(r){2-5} \cmidrule(r){6-9} \cmidrule(r){10-13}
			
			%			 subtitle
			& $E_{\phi}^{max}\uparrow$ & $S_{\alpha }\uparrow$ & $F_{\beta}^{max}\uparrow$ & MAE$\downarrow$ 
			& $E_{\phi}^{max}\uparrow$ & $S_{\alpha }\uparrow$ & $F_{\beta}^{max}\uparrow$ & MAE$\downarrow$  
			& $E_{\phi}^{max}\uparrow$ & $S_{\alpha }\uparrow$ & $F_{\beta}^{max}\uparrow$ & MAE$\downarrow$ \\
			
			%			& E & S & F & MAE 
			%			& E & S & F & MAE 
			%			& E & S & F & MAE \\
			
			
			% line line
			\midrule
			
			%			\multirow{8}*{\textit{Light field}}
			
			% 开始填数据
			
			Ours	 
			&  {\textbf{.973}} & \textbf{ {.946}} 	& \textbf{ {.954}} & \textbf{ {.020}} 
			& \textbf{ {.871}} &	\textbf{ {.828}} 			&\textbf{	 {.784}} & {\textcolor{red}{.064}} 
			& \textbf{ {.919}} &	\textcolor{blue}{.860} 			&	\textbf{ {.873}} &	\textbf{ {.064}} 
			\\
			
			DLGLRG \cite{liu2021light} 
			& {\textcolor{red}{.958}} & {\textcolor{red}{.928}} 			& {\textcolor{red}{.934}} & {\textcolor{red}{.029}} 
			&	.839 &	.766 &	.698 &	.070 
			&	{\textcolor{red}{.906}} &	\textbf{ {.866}} 			&	{\textcolor{red}{.870}} &	\textcolor{blue}{.069} 
			\\
			
			ERNet \cite{piao2020exploit}
			& .947 & .899 & .908 & .039 
			&	.841 &	.778 &	.722 &	.082 
			&	.888 &	.834 &	.850 &	.082 
			\\
			
			PANet \cite{piao2021panet} 
			& .939 & .908 & .903 & .038 
			& .845 & .795 & .738 & .074 
			& .892 & .849 & .849 & .076
			\\
			
			LFNet	 \cite{zhang2020lfnet} 
			& .929 & .878 & .890 & .053
			&	.846 &	.782 &	.718 &	.073 
			&	.885 &	.820 &	.824 &	.092 \\
			
			MAC	 \cite{zhang2020light} 
			& .863	& .804	& .792	& .102	
			&   .797 & .731 & .667 & .107 
			& .832 & .782 & .776 & .127 \\
			
			MoLF	 \cite{zhang2019memory} 
			& .938 & .887 & .902 & .051 
			&	.852 &	.789 &	.729 &	.075 
			&	.888 &	.830 &	.834 &	.089 \\
			
			DLSD	\cite{piao2019deep}
			& .891	& .841	& .801	& .076	
			&   .783 & .741 & .615 & .098 
			& .806 & .737 & .715 & .147 \\
			
			\midrule % end lfsod
			
			% start rgb-d
			%			\multirow{6}*{\textit{RGB-D}}
			
			DCF \cite{ji2021calibrated} 
			& \textcolor{blue}{.954} & \textcolor{blue}{.921} & \textcolor{blue}{.927} & \textcolor{blue}{.031} 
			& \textcolor{blue}{.856} & {\textcolor{red}{.812}} & {\textcolor{red}{.768}} & \textcolor{blue}{.065} 
			& .881 & .809 & .821 & .096 \\
			
			CIR-Net \cite{cong2022cir}
			& .950 & .916 & .921 & .038 
			& {\textcolor{red}{.862}} & .800  			& .742 & \textbf{ {.062}} 
			& .874 & .820 & .816 & .098 \\ 
			
			VST-$rgbd$  \cite{liu2021visual} 
			& .952 & .920 & .921 & .036 
			& .843 & .807 & .754 & .086 
			& .851 & .792 & .786 & .110 
			\\
			
			%			& -  & 2022  & \\
			%			& -  & 2022  & \\
			
			BBS-Net     \cite{fan2020bbs} 
			& .900 & .865 & .852 & .066 
			& .801 & .751 & .676 & .073 
			& \textcolor{blue}{.901} & {\textcolor{red}{.864}} & .858 & .072 \\ 
			
			SSF     \cite{zhang2020select} 
			& .922 & .879 & .887 & .050 
			& .816 & .725 & .647 & .090 
			& \textcolor{blue}{.901} & .859 & \textcolor{blue}{.868} & {\textcolor{red}{.067}} \\ 
			
			S2MA    \cite{liu2020learning} 
			& .839 & .787 & .754 & 	.102 
			& .777 & .729 & .650 & .112 
			& .873 & .837 &	.835 & .094 \\
			
			
			\midrule % end rgb-d
			%			\multirow{7}*{\textit{RGB}}
			
			VST-$rgb$ \cite{liu2021visual} 
			& .939 & .910 & .911 & .047
			& .831 & \textcolor{blue}{.808} & \textcolor{blue}{.763} & .093 
			& .865 & .797 & .817 & .123 
			\\ 
			
			PFSNet \cite{ma2021pyramidal}
			& .912 & .883 & .879 & .057 
			& .835 & .800 & .752 & .088 
			& .805 & .749 & .727 & .145 
			\\ 
			
			
			ITSD \cite{zhou2020interactive} 
			& .930 & .899 & .899 & .052 
			& .839 & .805 & .759 & .089 
			& .879 & .847 & .840 & .088 
			\\ 
			
			
			
			LDF \cite{wei2020label} 
			& .898 & .873 & .861 & .061 
			& .804 & .780 & .708 & .093 
			& .843 & .821 & .803 & .096 
			\\ 
			
			
			MINet \cite{pang2020multi} 
			& .916 & .890 & .882 & .050 
			& .816 & .792 & .720 & .086 
			& .861 & .834 & .828 & .091 
			\\ 
			
			F$^{3}$Net  \cite{wei2020f3net}
			& .900 & .888 & .882 & .057 
			& .815 & .777 & .718 & .095 
			& .824 & .806 & .797 & .106 
			\\ 
			
			
			EGNet   \cite{zhao2019egnet}
			& .914 & .886 & .870 & .053 
			& .794 & .772 & .672 & .094 
			& .776 & .784 & .762 & .118 
			\\ 
			
			CPD  \cite{wu2019cascaded}
			& .867 & .911 & .866 & .058 
			& .772 & .82  & .701 & .086 
			& .759 & .82  & .759 & .126 \\
			
			PoolNet \cite{liu2019simple}
			& .889 & .919 & .868 & .051 
			& .776 & .802 & .683 & .092 
			& .789 & .8   & .769 & .118 \\
			
			PiCANet \cite{liu2018picanet}
			& .829 & .892 & .821 & .083 
			& .726 & .781 & .618 & .115 
			& .729 & .78  & .671 & .158 \\
			
			PAGRN \cite{wang2018detect}
			& .822 & .878 & .828 & .084 
			& .717 & .773 & .635 & .114 
			& .727 & .805 & .725 & .147 \\
			
			C2S   \cite{li2018contour}
			& .844 & .874 & .791 & .084 
			& .763 & .786 & .65  & .111 
			& .806 & .82  & .749 & .113 \\
			
			R3Net  \cite{deng2018r3net}
			& .833 & .819 & .783 & .113 
			& .727 & .728 & .625 & .151 
			& .789 & .838 & .781 & .128 \\
			
			Amulet \cite{zhang2017amulet}
			& .847 & .882 & .805 & .030 
			& .767 & .76  & .636 & .110  
			& .773 & .821 & .757 & .135 \\
			
			
			\bottomrule % end
		\end{tabular}
	}
\end{table*}
%
%


为了进行全面比较,我们将我们的方法与 26 个最先进的模型进行比较,
包括6个光场显著性分割方法:
DLGLRG \cite{liu2021light}, RENet \cite{piao2020exploit}, LFNet \cite{zhang2020lfnet},
MAC \cite{zhang2020light}, MoLF \cite{zhang2019memory}, and DLSD \cite{piao2019deep};
%
%
%
%
6个RGB-D显著性分割方法:
DCF \cite{ji2021calibrated}, CIR-Net \cite{cong2022cir}, VST-$rgbd$  \cite{liu2021visual},
BBS-Net     \cite{fan2020bbs}, SSF     \cite{zhang2020select} and S2MA    \cite{liu2020learning};
%
%
%
%
%
和14个2D显著性检测方法:
VST-$rgb$ \cite{liu2021visual},
PFSNet \cite{ma2021pyramidal},
ITSD \cite{zhou2020interactive},
LDF \cite{wei2020label},
MINet \cite{pang2020multi},
F$^{3}$Net  \cite{wei2020f3net}, 
EGNet   \cite{zhao2019egnet},
CPD  \cite{wu2019cascaded},
PoolNet \cite{liu2019simple},
PiCANet \cite{liu2018picanet},
PAGRN \cite{wang2018detect},
C2S   \cite{li2018contour},
R3Net  \cite{deng2018r3net}
和
Amulet \cite{zhang2017amulet}。

为了保证公平比较,我们使用他们提供的显着性预测图或预训练的权重来生成比较数据,
并利用~\cite{liu2021visual}~中提供的相同评估代码。 
如表~\ref{chpt4:table:comp_with_sota_3_1}~所示,
很明显,所提出的方法在 DUTLF-FS 和 HFUT 数据集上实现了比当前最先进的方法更优越的性能。
%
%

\BiSection{本章总结}{TODO}

本章提出了一种视角增强的光场显著性目标检测方法。
此方法主要包含一个视角增强模块和一个像素对比学习策略,
通过端到端的网络训练得到显著性预测图。
具体来说:
(1)
%
%
本章方法所提出的视角增强注意力模块,
能够充分提取隐含在全聚焦图和焦点堆栈
内部的场景显著信息,并促进两种不同光场场景表示的融合。
(2)
%
%
提出的像素对比监督,
能够考虑显著区域和非显著区域的内在联系,
使得网络对于显著区域有更为鲁棒的辨别能力。
%
%
广泛的实验结果表明,
本章方法相比现有的光场显著性分割方法,具有更优越的显著性分割性能。









%
%
%\begin{table*}[]
%	%
%	%---------------------------------------------------------------------> 大表 
%	%
%	\caption{Quantitative comparison of our proposed FPT with other 20 SOTA SOD methods on three benchmark datasets. 
	%		$ \uparrow \& \downarrow $ denote larger and smaller is better.
	%		%
	%		% denote the best and the second-best results,
	%		%
	%		The best three results are shown in 
	%		\textbf{boldface}, \textcolor{red}{red} and \textcolor{blue}{blue} fonts respectively. 
	%		% '-' indicates the code or outcome is not available.
	%	}
%	\centering
%	\label{table:comp_with_sota_1}
%%	\resizebox{\textwidth}{!}{
	%		\begin{tabular}{crcccc}
		%			\toprule  %添加表格头部粗线
		%			
		%			% title
		%			\multirow{2}*{Type} & \multicolumn{1}{c}{ \multirow{2}*{Methods} } & 
		%			\multicolumn{4}{c}{DUTLF-FS \cite{zhang2019memory} } \\
		%
		%			
		%			% next line
		%			\cmidrule(r){3-6} 
		%			
		%			% subtitle
		%			& & 
		%			$E_{\phi}^{max}\uparrow$ & $S_{\alpha }\uparrow$ & $F_{\beta}^{max}\uparrow$ & MAE$\downarrow$ \\
		%		
		%			
		%			% line line
		%			\midrule
		%			
		%			\multirow{8}*{\textit{Light field}}
		%			
		%			% 开始填数据
		%			
		%			& Ours	 &  {\textbf{0.973}} & \textbf{ {0.946}} 
		%			& \textbf{ {0.954}} & \textbf{ {0.020}} 
		%			\\
		%			
		%			& DLGLRG \cite{liu2021light} 
		%			& {\textcolor{red}{0.958}} & {\textcolor{red}{0.928}} 
		%			& {\textcolor{red}{0.934}} & {\textcolor{red}{0.029}} 
		%			\\
		%			
		%			& ERNet \cite{piao2020exploit}
		%			& 0.947 & 0.899 & 0.908 & 0.039  \\
		%			
		%			& PANet \cite{piao2021panet} 
		%			% & 0.9390 & 0.9080 & 0.9029 & 0.0383 & 0.8449 & 0.7949 & 0.7383 & 0.0743 & 0.8922 & 0.8487 & 0.8494 & 0.0761 
		%			& 0.939 & 0.908 & 0.903 & 0.038
		%			\\
		%			
		%			& LFNet	 \cite{zhang2020lfnet} 
		%			& 0.929 & 0.878 & 0.890 & 0.053 	 \\
		%			
		%			& MAC	 \cite{zhang2020light} 
		%			& 0.863	& 0.804	& 0.792	& 0.102	    \\
		%			
		%			& MoLF	 \cite{zhang2019memory} 
		%			& 0.938 & 0.887 & 0.902 & 0.051 	 \\
		%			
		%			& DLSD	\cite{piao2019deep}
		%			& 0.891	& 0.841	& 0.801	& 0.076	    \\
		%			
		%			\midrule % end lfsod
		%			
		%			% start rgb-d
		%			\multirow{6}*{\textit{RGB-D}}
		%			
		%			% & 001& Male & 001& Male& 001& Male& 001& Male& 001& Male & 001& Male  & 001& Male \\
		%			
		%			& DCF \cite{ji2021calibrated} 
		%			& \textcolor{blue}{0.954} & \textcolor{blue}{0.921} & \textcolor{blue}{0.927} & \textcolor{blue}{0.031} 
		%		 	\\
		%			
		%			& CIR-Net \cite{cong2022cir}
		%			& 0.950 & 0.916 & 0.921 & 0.038 
		%		    \\ 
		%			
		%			& VST-$rgbd$  \cite{liu2021visual} 
		%			& 0.952 & 0.920 & 0.921 & 0.036 
		%			\\
		%			
		%			
		%			& BBS-Net     \cite{fan2020bbs} 
		%			& 0.900 & 0.865 & 0.852 & 0.066 
		%			\\ 
		%			
		%			& SSF     \cite{zhang2020select} 
		%			& 0.922 & 0.879 & 0.887 & 0.050 
		%			\\ 
		%			
		%			& S2MA    \cite{liu2020learning} 
		%			& 0.839 & 0.787 & 0.754 & 	0.102 
		%			\\
		%			
		%			\midrule % end rgb-d
		%			
		%			\multirow{7}*{\textit{RGB}}
		%			
		%			& VST-$rgb$ \cite{liu2021visual} 
		%			& 0.939 & 0.910 & 0.911 & 0.047  \\ 
		%			
		%			& PFSNet \cite{ma2021pyramidal}
		%			& 0.912 & 0.883 & 0.879 & 0.057  \\ 
		%			
		%			& ITSD \cite{zhou2020interactive} & 
		%			0.930 & 0.899 & 0.899 & 0.052  \\ 
		%			
		%			
		%			
		%			& LDF \cite{wei2020label} &
		%			0.898 & 0.873 & 0.861 & 0.061 \\ 
		%			
		%			
		%			& MINet \cite{pang2020multi} &
		%			0.916 & 0.890 & 0.882 & 0.050  \\ 
		%			
		%			& F$^{3}$Net  \cite{wei2020f3net}
		%			& 0.900 & 0.888 & 0.882 & 0.057  \\ 
		%			
		%			& EGNet   \cite{zhao2019egnet}
		%			& 0.914 & 0.886 & 0.870 & 0.053 \\ 
		%			
		%			\bottomrule % end
		%	\end{tabular}
	%%}
%\end{table*}
%
%
%
%\begin{table*}[]
%	%
%	%---------------------------------------------------------------------> 大表 
%	%
%	\caption{Quantitative comparison of our proposed FPT with other 20 SOTA SOD methods on three benchmark datasets. 
	%		$ \uparrow \& \downarrow $ denote larger and smaller is better.
	%		%
	%		% denote the best and the second-best results,
	%		%
	%		The best three results are shown in 
	%		\textbf{boldface}, \textcolor{red}{red} and \textcolor{blue}{blue} fonts respectively. 
	%		% '-' indicates the code or outcome is not available.
	%	}
%	\centering
%	\label{table:comp_with_sota_2}
%%	\label{table:comp-with-sota}
%%	\resizebox{\textwidth}{!}{
	%		\begin{tabular}{crcccc}
		%			\toprule  %添加表格头部粗线
		%			
		%			% title
		%			\multirow{2}*{Type} & \multicolumn{1}{c}{ \multirow{2}*{Methods} } & 
		%			\multicolumn{4}{c}{HFUT \cite{zhang2017saliency} } \\
		%			
		%			% next line
		%			\cmidrule(r){3-6} 
		%			
		%			% subtitle
		%			& & 
		%			$E_{\phi}^{max}\uparrow$ & $S_{\alpha }\uparrow$ & $F_{\beta}^{max}\uparrow$ & MAE$\downarrow$ \\
		%			
		%			% line line
		%			\midrule
		%			
		%			\multirow{8}*{\textit{Light field}}
		%			
		%			% 开始填数据
		%			
		%			& Ours	 
		%			& \textbf{ {0.871}} &	\textbf{ {0.828}} 
		%			&\textbf{	 {0.784}} & {\textcolor{red}{0.064}} 
		%		    \\
		%			
		%			& DLGLRG \cite{liu2021light} 
		%			&	0.839 &	0.766 &	0.698 &	0.070 
		%	        \\
		%			
		%			& ERNet \cite{piao2020exploit}
		%			&	0.841 &	0.778 &	0.722 &	0.082 \\
		%			
		%			& PANet \cite{piao2021panet} 
		%			% & 0.9390 & 0.9080 & 0.9029 & 0.0383 & 0.8449 & 0.7949 & 0.7383 & 0.0743 & 0.8922 & 0.8487 & 0.8494 & 0.0761 
		%			& 0.845 & 0.795 & 0.738 & 0.074 
		%			\\
		%			
		%			& LFNet	 \cite{zhang2020lfnet} 
		%			&	0.846 &	0.782 &	0.718 &	0.073 \\
		%			
		%			& MAC	 \cite{zhang2020light} 
		%			&   0.797 & 0.731 & 0.667 & 0.107 
		%			\\
		%			
		%			& MoLF	 \cite{zhang2019memory} 
		%			&	0.852 &	0.789 &	0.729 &	0.075 \\
		%			
		%			& DLSD	\cite{piao2019deep}
		%			&   0.783 & 0.741 & 0.615 & 0.098 \\
		%			
		%			
		%			\midrule % end lfsod
		%			
		%			\multirow{6}*{\textit{RGB-D}}
		%
		%			
		%			& DCF \cite{ji2021calibrated} 
		%			& \textcolor{blue}{0.856} & {\textcolor{red}{0.812}} & {\textcolor{red}{0.768}} & \textcolor{blue}{0.065} 
		%			\\
		%			
		%			& CIR-Net \cite{cong2022cir}
		%			& {\textcolor{red}{0.862}} & 0.800 
		%			& 0.742 & \textbf{ {0.062}} 
		%			\\ 
		%			
		%			& VST-$rgbd$  \cite{liu2021visual} 
		%			& 0.843 & 0.807 & 0.754 & 0.086  \\
		%
		%			
		%			& BBS-Net     \cite{fan2020bbs} 
		%			& 0.801 & 0.751 & 0.676 & 0.073 
		%			\\ 
		%			
		%			& SSF     \cite{zhang2020select} 
		%			& 0.816 & 0.725 & 0.647 & 0.090 
		%		    \\ 
		%			
		%			& S2MA    \cite{liu2020learning} 
		%		    & 0.777 & 0.729 & 0.650 & 0.112 \\
		%			
		%			
		%			%			& TriTransNet	&  & \\
		%			%			& DCFNet  & \\
		%			
		%			\midrule % end rgb-d
		%			
		%			% start rgb sod
		%			\multirow{7}*{\textit{RGB}}
		%			
		%			& VST-$rgb$ \cite{liu2021visual} 
		%			& 0.831 & \textcolor{blue}{0.808} & \textcolor{blue}{0.763} & 0.093 \\ 
		%			
		%			& PFSNet \cite{ma2021pyramidal}
		%			&  0.835 & 0.800 & 0.752 & 0.088  \\ 
		%			
		%			%			& - & \\
		%			
		%			& ITSD \cite{zhou2020interactive} & 
		%			 0.839 & 0.805 & 0.759 & 0.089  \\ 
		%			
		%			
		%			
		%			& LDF \cite{wei2020label} &
		%			 0.804 & 0.780 & 0.708 & 0.093  \\ 
		%			
		%			
		%			& MINet \cite{pang2020multi} 
		%			 & 0.816 & 0.792 & 0.720 & 0.086  \\ 
		%			
		%			& F$^{3}$Net  \cite{wei2020f3net}
		%			& 0.815 & 0.777 & 0.718 & 0.095  \\ 
		%
		%			
		%			& EGNet   \cite{zhao2019egnet}
		%			 & 0.794 & 0.772 & 0.672 & 0.094  \\ 
		%			
		%			\bottomrule % end
		%	\end{tabular}
	%%}
%\end{table*}


%
%\begin{table*}[]
%	%
%	%---------------------------------------------------------------------> 大表 
%	%
%	\caption{Quantitative comparison of our proposed FPT with other 20 SOTA SOD methods on three benchmark datasets. 
	%		$ \uparrow \& \downarrow $ denote larger and smaller is better.
	%		%
	%		% denote the best and the second-best results,
	%		%
	%		The best three results are shown in 
	%		\textbf{boldface}, \textcolor{red}{red} and \textcolor{blue}{blue} fonts respectively. 
	%		% '-' indicates the code or outcome is not available.
	%	}
%	\centering
%	\label{table:comp_with_sota_3}
%%	\label{table:comp-with-sota}
%%	\resizebox{\textwidth}{!}{
	%		\begin{tabular}{crcccc}
		%			\toprule  %添加表格头部粗线
		%			
		%%			\multirow{2}*{Type} & 
		%			\multicolumn{1}{c}{ \multirow{2}*{Methods} } & 
		%			% \multirow{2}*{Years} &
		%			% \multicolumn{1}{c}{Type} & \multicolumn{1}{c}{Methods} & \multicolumn{1}{c}{Years} & 
		%			\multicolumn{4}{c}{DUTLF-FS \cite{zhang2019memory} } &
		%			\multicolumn{4}{c}{HFUT \cite{zhang2017saliency} } &
		%			\multicolumn{4}{c}{LFSD \cite{li2014saliency} } \\
		%			
		%			% next line
		%			\cmidrule(r){2-5} \cmidrule(r){6-9} \cmidrule(r){10-13}
		%			
		%			% subtitle
		%			& 
		%			$E_{\phi}^{max}\uparrow$ & $S_{\alpha }\uparrow$ & $F_{\beta}^{max}\uparrow$ & MAE$\downarrow$ &
		%			$E_{\phi}^{max}\uparrow$ & $S_{\alpha }\uparrow$ & $F_{\beta}^{max}\uparrow$ & MAE$\downarrow$  &
		%			$E_{\phi}^{max}\uparrow$ & $S_{\alpha }\uparrow$ & $F_{\beta}^{max}\uparrow$ & MAE$\downarrow$ \\
		%			
		%			
		%			% line line
		%			\midrule
		%			
		%			\multirow{8}*{\textit{Light field}}
		%			
		%			% 开始填数据
		%			
		%			& Ours	 
		%%			&  {\textbf{0.973}} & \textbf{ {0.946}} 	& \textbf{ {0.954}} & \textbf{ {0.020}} 
		%%			& \textbf{ {0.871}} &	\textbf{ {0.828}} 			&\textbf{	 {0.784}} & {\textcolor{red}{0.064}} 
		%			& \textbf{ {0.919}} &	\textcolor{blue}{0.860} 			&	\textbf{ {0.873}} &	\textbf{ {0.064}} 
		%			\\
		%			
		%			& DLGLRG \cite{liu2021light} 
		%%			& {\textcolor{red}{0.958}} & {\textcolor{red}{0.928}} 			& {\textcolor{red}{0.934}} & {\textcolor{red}{0.029}} 
		%%			&	0.839 &	0.766 &	0.698 &	0.070 
		%			&	{\textcolor{red}{0.906}} &	\textbf{ {0.866}} 			&	{\textcolor{red}{0.870}} &	\textcolor{blue}{0.069} 
		%			\\
		%			
		%			& ERNet \cite{piao2020exploit}
		%%			& 0.947 & 0.899 & 0.908 & 0.039 
		%%			&	0.841 &	0.778 &	0.722 &	0.082 
		%			&	0.888 &	0.834 &	0.850 &	0.082 
		%			\\
		%			
		%			& PANet \cite{piao2021panet} 
		%%			& 0.939 & 0.908 & 0.903 & 0.038 
		%%			& 0.845 & 0.795 & 0.738 & 0.074 
		%			& 0.892 & 0.849 & 0.849 & 0.076
		%			\\
		%			
		%			& LFNet	 \cite{zhang2020lfnet} 
		%%			& 0.929 & 0.878 & 0.890 & 0.053
		%%			 &	0.846 &	0.782 &	0.718 &	0.073 
		%			 &	0.885 &	0.820 &	0.824 &	0.092 \\
		%			
		%			& MAC	 \cite{zhang2020light} 
		%%			& 0.863	& 0.804	& 0.792	& 0.102	
		%%			&   0.797 & 0.731 & 0.667 & 0.107 
		%			& 0.832 & 0.782 & 0.776 & 0.127 \\
		%			
		%			& MoLF	 \cite{zhang2019memory} 
		%%			& 0.938 & 0.887 & 0.902 & 0.051 
		%%			&	0.852 &	0.789 &	0.729 &	0.075 
		%			&	0.888 &	0.830 &	0.834 &	0.089 \\
		%			
		%			& DLSD	\cite{piao2019deep}
		%%			& 0.891	& 0.841	& 0.801	& 0.076	
		%%			&   0.783 & 0.741 & 0.615 & 0.098 
		%			& 0.806 & 0.737 & 0.715 & 0.147 \\
		%			
		%			\midrule % end lfsod
		%			
		%			% start rgb-d
		%			\multirow{6}*{\textit{RGB-D}}
		%			
		%			& DCF \cite{ji2021calibrated} 
		%%			& \textcolor{blue}{0.954} & \textcolor{blue}{0.921} & \textcolor{blue}{0.927} & \textcolor{blue}{0.031} 
		%%			& \textcolor{blue}{0.856} & {\textcolor{red}{0.812}} & {\textcolor{red}{0.768}} & \textcolor{blue}{0.065} 
		%			& 0.881 & 0.809 & 0.821 & 0.096 \\
		%			
		%			& CIR-Net \cite{cong2022cir}
		%%			& 0.950 & 0.916 & 0.921 & 0.038 
		%%			& {\textcolor{red}{0.862}} & 0.800  			& 0.742 & \textbf{ {0.062}} 
		%			& 0.874 & 0.820 & 0.816 & 0.098 \\ 
		%			
		%			& VST-$rgbd$  \cite{liu2021visual} 
		%%			& 0.952 & 0.920 & 0.921 & 0.036 
		%%			& 0.843 & 0.807 & 0.754 & 0.086 
		%			& 0.851 & 0.792 & 0.786 & 0.110 
		%			\\
		%			
		%			%			& -  & 2022  & \\
		%			%			& -  & 2022  & \\
		%			
		%			& BBS-Net     \cite{fan2020bbs} 
		%%			& 0.900 & 0.865 & 0.852 & 0.066 
		%%			& 0.801 & 0.751 & 0.676 & 0.073 
		%			& \textcolor{blue}{0.901} & {\textcolor{red}{0.864}} & 0.858 & 0.072 \\ 
		%			
		%			& SSF     \cite{zhang2020select} 
		%%			& 0.922 & 0.879 & 0.887 & 0.050 
		%%			& 0.816 & 0.725 & 0.647 & 0.090 
		%			& \textcolor{blue}{0.901} & 0.859 & \textcolor{blue}{0.868} & {\textcolor{red}{0.067}} \\ 
		%			
		%			& S2MA    \cite{liu2020learning} 
		%%			& 0.839 & 0.787 & 0.754 & 	0.102 
		%%			& 0.777 & 0.729 & 0.650 & 0.112 
		%			& 0.873 & 0.837 &	0.835 & 0.094 \\
		%
		%			
		%			\midrule % end rgb-d
		%			\multirow{7}*{\textit{RGB}}
		%			
		%			& VST-$rgb$ \cite{liu2021visual} 
		%%			& 0.939 & 0.910 & 0.911 & 0.047
		%%			& 0.831 & \textcolor{blue}{0.808} & \textcolor{blue}{0.763} & 0.093 
		%			& 0.865 & 0.797 & 0.817 & 0.123 
		%			\\ 
		%			
		%			& PFSNet \cite{ma2021pyramidal}
		%%			& 0.912 & 0.883 & 0.879 & 0.057 
		%%			& 0.835 & 0.800 & 0.752 & 0.088 
		%			& 0.805 & 0.749 & 0.727 & 0.145 
		%			\\ 
		%
		%			
		%			& ITSD \cite{zhou2020interactive} 
		%%			& 0.930 & 0.899 & 0.899 & 0.052 
		%%			& 0.839 & 0.805 & 0.759 & 0.089 
		%			& 0.879 & 0.847 & 0.840 & 0.088 
		%			\\ 
		%			
		%			
		%			
		%			& LDF \cite{wei2020label} 
		%%			& 0.898 & 0.873 & 0.861 & 0.061 
		%%			& 0.804 & 0.780 & 0.708 & 0.093 
		%			& 0.843 & 0.821 & 0.803 & 0.096 
		%			\\ 
		%			
		%			
		%			& MINet \cite{pang2020multi} 
		%%			& 0.916 & 0.890 & 0.882 & 0.050 
		%%			& 0.816 & 0.792 & 0.720 & 0.086 
		%			& 0.861 & 0.834 & 0.828 & 0.091 
		%			\\ 
		%			
		%			& F$^{3}$Net  \cite{wei2020f3net}
		%%			& 0.900 & 0.888 & 0.882 & 0.057 
		%%			& 0.815 & 0.777 & 0.718 & 0.095 
		%			& 0.824 & 0.806 & 0.797 & 0.106 
		%			\\ 
		%			
		%			
		%			& EGNet   \cite{zhao2019egnet}
		%%			& 0.914 & 0.886 & 0.870 & 0.053 
		%%			& 0.794 & 0.772 & 0.672 & 0.094 
		%			& 0.776 & 0.784 & 0.762 & 0.118 
		%			\\ 
		%			
		%			\bottomrule % end
		%	\end{tabular}
	%%}
%\end{table*}




%
%\begin{table*}[]
%	%
%	%---------------------------------------------------------------------> 大表 
%	%
%	\caption{Quantitative comparison of our proposed FPT with other 20 SOTA SOD methods on three benchmark datasets. 
	%		$ \uparrow \& \downarrow $ denote larger and smaller is better.
	%		%
	%		% denote the best and the second-best results,
	%		%
	%		The best three results are shown in 
	%		\textbf{boldface}, \textcolor{red}{red} and \textcolor{blue}{blue} fonts respectively. 
	%		% '-' indicates the code or outcome is not available.
	%	}
%	\centering
%	\label{table:comp_with_sota_3}
%	%	\label{table:comp-with-sota}
%		\resizebox{\textwidth}{!}{
	%		\begin{tabular}{rcccccccccccc}
		%			\toprule  %添加表格头部粗线
		%			
		%			% title
		%%			\multirow{2}*{Type} & 
		%			\multicolumn{1}{c}{ \multirow{2}*{Methods} } & 
		%			\multicolumn{4}{c}{DUTLF-FS \cite{zhang2019memory} } &
		%			\multicolumn{4}{c}{HFUT \cite{zhang2017saliency} } &
		%			\multicolumn{4}{c}{LFSD \cite{li2014saliency} } \\
		%			
		%			% next line
		%			\cmidrule(r){2-5} \cmidrule(r){6-9} \cmidrule(r){10-13}
		%			
		%			% subtitle
		%			& 
		%			$E_{\phi}^{max}\uparrow$ & $S_{\alpha }\uparrow$ & $F_{\beta}^{max}\uparrow$ & MAE$\downarrow$ &
		%			$E_{\phi}^{max}\uparrow$ & $S_{\alpha }\uparrow$ & $F_{\beta}^{max}\uparrow$ & MAE$\downarrow$  &
		%			$E_{\phi}^{max}\uparrow$ & $S_{\alpha }\uparrow$ & $F_{\beta}^{max}\uparrow$ & MAE$\downarrow$ \\
		%			
		%			
		%			
		%			% line line
		%			\midrule
		%			
		%%			\multirow{8}*{\textit{Light field}}
		%			
		%			% 开始填数据
		%			
		%			 Ours	 
		%						&  {\textbf{0.973}} & \textbf{ {0.946}} 	& \textbf{ {0.954}} & \textbf{ {0.020}} 
		%						& \textbf{ {0.871}} &	\textbf{ {0.828}} 			&\textbf{	 {0.784}} & {\textcolor{red}{0.064}} 
		%			& \textbf{ {0.919}} &	\textcolor{blue}{0.860} 			&	\textbf{ {0.873}} &	\textbf{ {0.064}} 
		%			\\
		%			
		%			 DLGLRG \cite{liu2021light} 
		%						& {\textcolor{red}{0.958}} & {\textcolor{red}{0.928}} 			& {\textcolor{red}{0.934}} & {\textcolor{red}{0.029}} 
		%						&	0.839 &	0.766 &	0.698 &	0.070 
		%			&	{\textcolor{red}{0.906}} &	\textbf{ {0.866}} 			&	{\textcolor{red}{0.870}} &	\textcolor{blue}{0.069} 
		%			\\
		%			
		%			 ERNet \cite{piao2020exploit}
		%						& 0.947 & 0.899 & 0.908 & 0.039 
		%						&	0.841 &	0.778 &	0.722 &	0.082 
		%			&	0.888 &	0.834 &	0.850 &	0.082 
		%			\\
		%			
		%			 PANet \cite{piao2021panet} 
		%						& 0.939 & 0.908 & 0.903 & 0.038 
		%						& 0.845 & 0.795 & 0.738 & 0.074 
		%			& 0.892 & 0.849 & 0.849 & 0.076
		%			\\
		%			
		%			 LFNet	 \cite{zhang2020lfnet} 
		%						& 0.929 & 0.878 & 0.890 & 0.053
		%						 &	0.846 &	0.782 &	0.718 &	0.073 
		%			&	0.885 &	0.820 &	0.824 &	0.092 \\
		%			
		%			 MAC	 \cite{zhang2020light} 
		%						& 0.863	& 0.804	& 0.792	& 0.102	
		%						&   0.797 & 0.731 & 0.667 & 0.107 
		%			& 0.832 & 0.782 & 0.776 & 0.127 \\
		%			
		%			 MoLF	 \cite{zhang2019memory} 
		%						& 0.938 & 0.887 & 0.902 & 0.051 
		%						&	0.852 &	0.789 &	0.729 &	0.075 
		%			&	0.888 &	0.830 &	0.834 &	0.089 \\
		%			
		%			 DLSD	\cite{piao2019deep}
		%						& 0.891	& 0.841	& 0.801	& 0.076	
		%						&   0.783 & 0.741 & 0.615 & 0.098 
		%			& 0.806 & 0.737 & 0.715 & 0.147 \\
		%			
		%			\midrule % end lfsod
		%			
		%			% start rgb-d
		%%			\multirow{6}*{\textit{RGB-D}}
		%			
		%			 DCF \cite{ji2021calibrated} 
		%						& \textcolor{blue}{0.954} & \textcolor{blue}{0.921} & \textcolor{blue}{0.927} & \textcolor{blue}{0.031} 
		%						& \textcolor{blue}{0.856} & {\textcolor{red}{0.812}} & {\textcolor{red}{0.768}} & \textcolor{blue}{0.065} 
		%			& 0.881 & 0.809 & 0.821 & 0.096 \\
		%			
		%			 CIR-Net \cite{cong2022cir}
		%						& 0.950 & 0.916 & 0.921 & 0.038 
		%						& {\textcolor{red}{0.862}} & 0.800  			& 0.742 & \textbf{ {0.062}} 
		%			& 0.874 & 0.820 & 0.816 & 0.098 \\ 
		%			
		%		 VST-$rgbd$  \cite{liu2021visual} 
		%						& 0.952 & 0.920 & 0.921 & 0.036 
		%						& 0.843 & 0.807 & 0.754 & 0.086 
		%			& 0.851 & 0.792 & 0.786 & 0.110 
		%			\\
		%			
		%			%			& -  & 2022  & \\
		%			%			& -  & 2022  & \\
		%			
		%			 BBS-Net     \cite{fan2020bbs} 
		%						& 0.900 & 0.865 & 0.852 & 0.066 
		%						& 0.801 & 0.751 & 0.676 & 0.073 
		%			& \textcolor{blue}{0.901} & {\textcolor{red}{0.864}} & 0.858 & 0.072 \\ 
		%			
		%			 SSF     \cite{zhang2020select} 
		%						& 0.922 & 0.879 & 0.887 & 0.050 
		%						& 0.816 & 0.725 & 0.647 & 0.090 
		%			& \textcolor{blue}{0.901} & 0.859 & \textcolor{blue}{0.868} & {\textcolor{red}{0.067}} \\ 
		%			
		%			 S2MA    \cite{liu2020learning} 
		%						& 0.839 & 0.787 & 0.754 & 	0.102 
		%						& 0.777 & 0.729 & 0.650 & 0.112 
		%			& 0.873 & 0.837 &	0.835 & 0.094 \\
		%			
		%			
		%			\midrule % end rgb-d
		%%			\multirow{7}*{\textit{RGB}}
		%			
		%			 VST-$rgb$ \cite{liu2021visual} 
		%						& 0.939 & 0.910 & 0.911 & 0.047
		%						& 0.831 & \textcolor{blue}{0.808} & \textcolor{blue}{0.763} & 0.093 
		%			& 0.865 & 0.797 & 0.817 & 0.123 
		%			\\ 
		%			
		%			 PFSNet \cite{ma2021pyramidal}
		%						& 0.912 & 0.883 & 0.879 & 0.057 
		%						& 0.835 & 0.800 & 0.752 & 0.088 
		%			& 0.805 & 0.749 & 0.727 & 0.145 
		%			\\ 
		%			
		%			
		%			 ITSD \cite{zhou2020interactive} 
		%						& 0.930 & 0.899 & 0.899 & 0.052 
		%						& 0.839 & 0.805 & 0.759 & 0.089 
		%			& 0.879 & 0.847 & 0.840 & 0.088 
		%			\\ 
		%			
		%			
		%			
		%			LDF \cite{wei2020label} 
		%						& 0.898 & 0.873 & 0.861 & 0.061 
		%						& 0.804 & 0.780 & 0.708 & 0.093 
		%			& 0.843 & 0.821 & 0.803 & 0.096 
		%			\\ 
		%			
		%			
		%			MINet \cite{pang2020multi} 
		%						& 0.916 & 0.890 & 0.882 & 0.050 
		%						& 0.816 & 0.792 & 0.720 & 0.086 
		%			& 0.861 & 0.834 & 0.828 & 0.091 
		%			\\ 
		%			
		%			F$^{3}$Net  \cite{wei2020f3net}
		%						& 0.900 & 0.888 & 0.882 & 0.057 
		%						& 0.815 & 0.777 & 0.718 & 0.095 
		%			& 0.824 & 0.806 & 0.797 & 0.106 
		%			\\ 
		%			
		%			
		%			EGNet   \cite{zhao2019egnet}
		%						& 0.914 & 0.886 & 0.870 & 0.053 
		%						& 0.794 & 0.772 & 0.672 & 0.094 
		%			& 0.776 & 0.784 & 0.762 & 0.118 
		%			\\ 
		%			
		%			\bottomrule % end
		%		\end{tabular}
	%		}
%\end{table*}

%%\BiChapter{结论与展望}{Conclusions and Prospection}
%该部分主要包括三个部分:“结论”、“创新点”和“展望”。
%\BiSection{结论}{ Conclusions}
%结论是理论分析和实验结果的逻辑发展,是整篇论文的归宿。结论是在理论分析、试验结果的基础上,经过分析、推理、判断、归纳的过程而形成的总观点。结论必须完整、准确、鲜明、并突出与前人不同的新见解。
%\BiSection{创新点}{Highlights}
%创新点应该以分条列举的形式进行提出。\par
%(1) 以预报……模型,建立了….。 \par
%(2) 应用……方法,对颗粒受力情况进行了分析。\par
%(3) ……\par
%(4) ……
%\BiSection{展望}{Prospection}
%展望是对该研究课题存在的不足和有待改进的说明,是对未来研究的一种期待。


\BiChapter{结论与展望}{Conclusions and Prospection}


%该部分主要包括三个部分:“结论”、“创新点”和“展望”。


\BiSection{结论}{ Conclusions}
%
%
%结论是理论分析和实验结果的逻辑发展,是整篇论文的归宿。结论是在理论分析、试验结果的基础上,经过分析、推理、判断、归纳的过程而形成的总观点。结论必须完整、准确、鲜明、并突出与前人不同的新见解。
%
%
%
%然而,在实际场景中,光场的获取成本高、多线索信息处理复杂,以及耗时费力的像素级显著性标注,导致当前光场显著性目标检测数据稀缺,难以支撑深度模型训练所需的数据。针对上述挑战,本文通过高效利用光场信息和增加光场数据量两方面入手,探索了利用有限数据进行光场显著性目标检测的方法。主要研究内容如下:
%
%






光场显著目标检测旨在从周围环境中分割出视觉上独特的对象。
不同于只在彩色图像上的RGB显著性目标检测和
在彩色图像上用深度信息辅助的RGB-D显著性目标检测不同,
光场图像提供多焦点堆栈(不同深度级别的多个焦段)和同一场景的全焦点图像,
它们记录了全面但冗余的信息。
然而,实际应用中存在复杂的光场信息提取以及跨模态的光场信息融合难等问题,
这导致了当前光场显著性检测深度模型难以有效辨别光场场景的的显著性物体表示。
为了解决这些问题,本文从焦点感知和视角增强两个角度出发,
探索基于聚焦感知的光场显著性检测方法。
本文的主要工作如下:




%%%%%%%%%%%%%%%%%%%%%%%%%%%%%%%%%%%%%%%%%%%%%
%
% 第一个工作点
%
%%%%%%%%%%%%%%%%%%%%%%%%%%%%%%%%%%%%%%%%%%%%%
(1)
%
%
面对如何有效利用复杂场景中丰富的光场线索的挑战,
本文提出了一种聚焦感知网络探索光场数据的方法。
该方法主要包含两个部分:令牌通信模块和聚焦感知增强策略。
其中令牌通信模块通过嵌入式令牌表示汇总建立全聚焦图片和焦点堆栈的切片级特征,
并通过令牌作为信息传递的桥梁,促进网络对空间上下文建模。
聚焦感知增强策略充分考虑不同聚焦切片对于显著性的影响,
通过判断每个散焦切片的聚焦程度,来突出不同焦点切片中
显著性区域,同时抑制非显著性区域带来的负面影响。
相比现有的方法,本文方法通过附加嵌入式令牌的方式,
对光场的整体三维场景进行了切片级的探索,
并考虑了不同散焦切片对显著性预测的贡献,
能够更有效的利用光场信息。

%%%%%%%%%%%%%%%%%%%%%%%%%%%%%%%%%%%%%%%%%%%%%
%
% 第二个工作点
%
%%%%%%%%%%%%%%%%%%%%%%%%%%%%%%%%%%%%%%%%%%%%%
(2)
%
%
面对如何高效的利用光场数据中全聚焦图和焦点堆栈两个模态的差异信息,
本文提出了一种视角增强网络探索光场数据的方法。
该方法主要包含两个部分:视角增强注意力模块和感知对比学习策略。
其中视角增强注意力模块通过对两个模态做交叉注意力时引入跨模态的掩码表达,
加强了注意力权重在不同聚焦区域上的显著性表达。
感知对比学习策略考虑显著性预测的前景区域内部,与背景区域内部的一致性表达。
相比现有的光场显著性检测方法,本文方法对光场数据进行跨模态的特征融合,
充分考虑了焦点堆栈和全聚焦图对最终显著性预测的贡献,
能够产生更为鲁棒的显著性物体表达。




\BiSection{创新点}{Highlights}
%
%
%创新点应该以分条列举的形式进行提出。\par
%(1) 以预报……模型,建立了….。 \par
%(2) 应用……方法,对颗粒受力情况进行了分析。\par
%(3) ……\par
%(4) ……
%
%


(1)
通过构建切片级嵌入式令牌表示,通过令牌之间的交叉注意力和移位操作,
建立了网络对于光场三维场景的感知。


(2)
通过建立全聚焦特征与焦点堆栈特征的聚焦聚焦匹配,
增强了焦点堆栈中显著物体表示,并抑制非显著的背景噪声影响。


(3)
提出视角增强注意力来促进跨焦点堆栈和全聚焦特征的融合,
在进行两个模态的交叉注意计算时,使用跨模态的显著性前景表达来增强权重矩阵,
降低了由于焦点堆栈中背景聚焦引起的注意力转移。


(4)
引入像素对比学习策略,使得网络在分辨像素类别的同时,也能够考虑
显著性像素区域的内在联系和与非显著性区域的整体差异。
增加了网络对显著性区域的辨识能力。



\BiSection{展望}{Prospection}
%
%展望是对该研究课题存在的不足和有待改进的说明,是对未来研究的一种期待。
%
%

得益于光场信息独有的高维数据表示,
光场显著性目标检测网络在一些复杂场景上相比RGB或RGB-D的
显著性目标检测具有更大的优势。
但是,光场的多模态数据表示,需要使用多支路的神经网络进行差异化建模,
使得网络必须具有较高的数据吞吐量和足够的网络深度,
相应也带来了更高的计算负担。
现阶段的光场显著性目标检测网络大都没有实时处理的能力。
这限制了光场数据的应用。

本文将在接下来的工作中,探索使用轻量化的网络实现光场显著性目标检测。
目标是在较小网络参数量和较低计算量的基础上,实现性能优异的光场显著性目标检测网络。
同时,现阶段的光场显著性目标检测多是在离线数据集上进行训练和测试,
本文设想构造端到端的光场显著性网络,即在轻量化网络基础上,
用一个数据流实现从光场相机的数据采集与显著性目标检测。























%\include{chapters/chapter1}
%\include{chapters/chapter2}
%\include{chapters/chapter3}
%\include{chapters/chapter4}
%\include{chapters/chapter5}
%\include{chapters/chapter6}


%% 参考文献,五号字,使用 BibTeX,包含参考文献文件.bib
%\bibliography{reference/chap1,reference/chap2} %多个章节的参考文献
\bibliography{reference/lfsod,reference/chap1,reference/ref_chapter2,reference/lfsod_models}


% %%%%%%%%%%%%%%%%%%%%%%%%%%%%%%
% %% 后置部分
% %%%%%%%%%%%%%%%%%%%%%%%%%%%%%%

% %% 附录(章节编号重新计算,使用字母进行编号)
% \appendix
% \renewcommand{\appendixname}{附录~\Alph{chapter}}
% \CTEXsetup[name={附录}]{chapter}
%  % 附录中编号形式是"A.1"的样子
% \renewcommand\thefigure{\Alph{chapter}.\arabic{figure}}
% \renewcommand\thetable{\Alph{chapter}.\arabic{table}}
% \renewcommand{\theequation}{\Alph{chapter}.\arabic{equation}}
% \renewcommand{\thelstlisting}{\Alph{chapter}.\arabic{lstlisting}}
% \renewcommand\tablename{附录-表}
% \captionsetup[table][bi-second]{name=App.Tab.}
% \renewcommand\figurename{附录-图}
% \captionsetup[figure][bi-second]{name=App.Fig.}

%\include{chapters/app1} 
%\include{chapters/app2} 

% %(其后部分无编号)
%\backmatter  

% % 发表文章目录
%\include{chapters/pub}
%\include{parts/pub}

% % 致谢
%%%==================================================
%% thanks.tex for DUT Thesis
%% version: 0.1
%% last update: Apr 27th, 2022
%%==================================================

\begin{thanks}
%学位论文中不得书写与论文工作无关的人和事(可以写家人),对导师的致谢要实事求是。\par
%对指导或协助指导完成论文的导师、资助基金或合同单位、提供帮助和支持的同志应在论文中做明确的说明并表示谢意。\par
%这部分内容不可省略。\par
落笔至此,感慨万千。三年如白驹过隙,眼下虽是论文的终章,却是未来的起点。
回首这几年,我曾憧憬过、迷茫过、失落过、拼搏过,经历了诸多起伏,收获了成长与收获。
在此,我要衷心感谢所有曾经帮助过我的人,也感谢自己的努力和坚持。
\\
%
%
%
%
\indent
首先,特别感谢我的导师朴永日副教授。
朴老师的一丝不苟、开拓创新的科研思维和积极乐观态度深深激励与教导了我。
感谢朴老师在科研上的细致指导,帮助我规划研究方向,培养我的科研能力。
也感谢朴老师对我生活的关怀,在我失落时,朴老师给予了很多勉励和支持,让我重新振作面对困难和挫折。
同时,也深表感谢同组的张淼副教授,感谢她提供的写作思路和修改建议,
张老师的学术严谨与精益求精的工作作风让我受益匪浅。
两位导师是我读研期间最大的恩人,两位导师的教诲我将永铭心怀,能成为他们的学生,是我毕生的幸运。
\\
%
%
%
%
\indent
其次,感谢所有关心帮助过我的同学们,能结识课题组中的同学是我一生中最大的财富。
感谢姜永耀师兄、王健师兄和吴为师兄在我初踏入实验室时的耐心指导;
感谢赵永帅师兄,陆晨阳师兄和刘廷位师兄在科研和工作上对我的支持;
感谢孙小飞师兄,许爽师姐对我实习工作上的帮助,缓解了我初入职场的忐忑,
让我更快的适应了工作环境;
感谢刘垒烨和尹纪浩对我竞赛上的帮助,不仅让我学到了很多新知识,也让我感受到团队的力量!
同时也祝愿姚臻彦、王治、
王书{\CJKfontspec{NotoSerifTC-Regular.otf} 垚}、
钟嘉龙、马宁等师弟师妹们在科研之路上一帆风顺。
感谢同门吴岚虎和李智玮,至今都在怀念咱们一起探讨科研的时光,
感谢你们的帮助,我们来日再相聚。
\\
%
%
%
%
\indent
最后,由衷感谢我的家人对我所给予的支持,没有他们,我不可能走到今天。我也要感谢当初选择考研时的自己,给了未来自己另一种可能性;感谢在考研过程中我能够坚持孤独、夜以继日地学习;更要感谢在研究生阶段我能够保持坚持并且有所收获。未来将有更广阔的世界等待着我,希望我能时刻记住最初的目标,不忘初心,继续前行。
\\
%
%
%
%
\indent
从家到学校,短短二十公里之路,走过整整二十年。如今即将走向江湖,将来漫漫云霄,我们各自有着不同的归程!祝愿国家稳步复兴、富强兴盛,也希望自己一帆风顺、幸福安康!再次感谢所有关心帮助过我的人,我将继续努力,成为更优秀的自己。
%{\CJKfontspec{NotoSerifTC-Regular.otf} 想要修改字体的部分
%
%鑫
%犇
%焱
%懿
%燚
%垚
%}
\end{thanks}


%%%==================================================
%% thanks.tex for DUT Thesis
%% version: 0.1
%% last update: Apr 27th, 2022
%%==================================================

\begin{thanks}
%学位论文中不得书写与论文工作无关的人和事(可以写家人),对导师的致谢要实事求是。\par
%对指导或协助指导完成论文的导师、资助基金或合同单位、提供帮助和支持的同志应在论文中做明确的说明并表示谢意。\par
%这部分内容不可省略。\par
落笔至此,感慨万千。三年如白驹过隙,眼下虽是论文的终章,却是未来的起点。
回首这几年,我曾憧憬过、迷茫过、失落过、拼搏过,经历了诸多起伏,收获了成长与收获。
在此,我要衷心感谢所有曾经帮助过我的人,也感谢自己的努力和坚持。
\\
%
%
%
%
\indent
首先,特别感谢我的导师朴永日副教授。
朴老师的一丝不苟、开拓创新的科研思维和积极乐观态度深深激励与教导了我。
感谢朴老师在科研上的细致指导,帮助我规划研究方向,培养我的科研能力。
也感谢朴老师对我生活的关怀,在我失落时,朴老师给予了很多勉励和支持,让我重新振作面对困难和挫折。
同时,也深表感谢同组的张淼副教授,感谢她提供的写作思路和修改建议,
张老师的学术严谨与精益求精的工作作风让我受益匪浅。
两位导师是我读研期间最大的恩人,两位导师的教诲我将永铭心怀,能成为他们的学生,是我毕生的幸运。
\\
%
%
%
%
\indent
其次,感谢所有关心帮助过我的同学们,能结识课题组中的同学是我一生中最大的财富。
感谢姜永耀师兄、王健师兄和吴为师兄在我初踏入实验室时的耐心指导;
感谢赵永帅师兄,陆晨阳师兄和刘廷位师兄在科研和工作上对我的支持;
感谢孙小飞师兄,许爽师姐对我实习工作上的帮助,缓解了我初入职场的忐忑,
让我更快的适应了工作环境;
感谢刘垒烨和尹纪浩对我竞赛上的帮助,不仅让我学到了很多新知识,也让我感受到团队的力量!
同时也祝愿姚臻彦、王治、
王书{\CJKfontspec{NotoSerifTC-Regular.otf} 垚}、
钟嘉龙、马宁等师弟师妹们在科研之路上一帆风顺。
感谢同门吴岚虎和李智玮,至今都在怀念咱们一起探讨科研的时光,
感谢你们的帮助,我们来日再相聚。
\\
%
%
%
%
\indent
最后,由衷感谢我的家人对我所给予的支持,没有他们,我不可能走到今天。我也要感谢当初选择考研时的自己,给了未来自己另一种可能性;感谢在考研过程中我能够坚持孤独、夜以继日地学习;更要感谢在研究生阶段我能够保持坚持并且有所收获。未来将有更广阔的世界等待着我,希望我能时刻记住最初的目标,不忘初心,继续前行。
\\
%
%
%
%
\indent
从家到学校,短短二十公里之路,走过整整二十年。如今即将走向江湖,将来漫漫云霄,我们各自有着不同的归程!祝愿国家稳步复兴、富强兴盛,也希望自己一帆风顺、幸福安康!再次感谢所有关心帮助过我的人,我将继续努力,成为更优秀的自己。
%{\CJKfontspec{NotoSerifTC-Regular.otf} 想要修改字体的部分
%
%鑫
%犇
%焱
%懿
%燚
%垚
%}
\end{thanks}




\end{document}
