\BiChapter{Latex模板简介}{The Introduction of Latex Template}
本章介绍Latex的文本控制流程,介绍学位论文中各章节分布及模板内部组成,用户可根据自身对Latex的熟悉程度适当地略过阅读。\par
\BiSection{认识模板组成}{Understanding Template Composition}
格式控制文件控制着论文的表现形式,包括以下两个后缀的文件:“.cls”后缀的文件控制论文主体格式和“.bst”后缀的文件控制参考文献条目。\par
一般用户在.cls后缀的文件里填写封面、申明和授权书及奇偶页页眉内容,然后在各章正文处进行内容的填写。
\BiSection{主控文件}{Main File}
主控文件MA Thesis.tex的作用就是将分散在多个文件中的内容整合成一篇完整的内容。此部分用户不需做改动。
\BiSection{论文主体文件夹chapters}{Chapters File}
这一部分是论文的主体,是以“章”为单位划分的。用户需要在这里对中英文摘要、符号表、各章内容、附录、致谢、攻读学位论文期间的成果及作者简介进行填写。
\BiSection{图片文件夹figures}{Figures File}
figures文件夹放置了需要插入文档中的图片文件(PNG/JPG/PDF/EPS)。如果图片较多,建议按章再划分子目录存储图片。
\BiSection{参考文献数据库文件夹reference}{Reference File}
此文件夹用于存放参考文献信息。bib数据库中的参考文献条目可以手动编写,也可以在Google的学术搜索中找到。各大数据库也支持将参考文献信息导出为.bib。
\BiSection{本章小结}{The Chapter’s Conclusion}



