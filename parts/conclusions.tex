%\BiChapter{结论与展望}{Conclusions and Prospection}
%该部分主要包括三个部分:“结论”、“创新点”和“展望”。
%\BiSection{结论}{ Conclusions}
%结论是理论分析和实验结果的逻辑发展,是整篇论文的归宿。结论是在理论分析、试验结果的基础上,经过分析、推理、判断、归纳的过程而形成的总观点。结论必须完整、准确、鲜明、并突出与前人不同的新见解。
%\BiSection{创新点}{Highlights}
%创新点应该以分条列举的形式进行提出。\par
%(1) 以预报……模型,建立了….。 \par
%(2) 应用……方法,对颗粒受力情况进行了分析。\par
%(3) ……\par
%(4) ……
%\BiSection{展望}{Prospection}
%展望是对该研究课题存在的不足和有待改进的说明,是对未来研究的一种期待。


\BiChapter{结论与展望}{Conclusions and Prospection}
%该部分主要包括三个部分:“结论”、“创新点”和“展望”。
\BiSection{结论}{ Conclusions}
%
%
%结论是理论分析和实验结果的逻辑发展,是整篇论文的归宿。结论是在理论分析、试验结果的基础上,经过分析、推理、判断、归纳的过程而形成的总观点。结论必须完整、准确、鲜明、并突出与前人不同的新见解。
%
%
%
%然而,在实际场景中,光场的获取成本高、多线索信息处理复杂,以及耗时费力的像素级显著性标注,导致当前光场显著性目标检测数据稀缺,难以支撑深度模型训练所需的数据。针对上述挑战,本文通过高效利用光场信息和增加光场数据量两方面入手,探索了利用有限数据进行光场显著性目标检测的方法。主要研究内容如下:
%
%
光场显著目标检测旨在从周围环境中分割出视觉上独特的对象。
不同于只在彩色图像上的RGB显著性目标检测和
在彩色图像上用深度信息辅助的RGB-D显著性目标检测不同,
光场图像提供多焦点堆栈(不同深度级别的多个焦段)和同一场景的全焦点图像,
它们记录了全面但冗余的信息。
然而,实际应用中存在复杂的光场信息提取以及跨模态的光场信息融合难等问题,
这导致了当前光场显著性检测深度模型难以有效辨别光场场景的的显著性物体表示。
为了解决这些问题,本文从焦点感知和视角增强两个角度出发,
探索基于聚焦感知的光场显著性检测方法。
本文的主要工作如下:
\\
%
%
%
%
\indent
%%%%%%%%%%%%%%%%%%%%%%%%%%%%%%%%%%%%%%%%%%%%%
%
% 第一个工作点
%
%%%%%%%%%%%%%%%%%%%%%%%%%%%%%%%%%%%%%%%%%%%%%
(1)
%
%
面对如何有效利用复杂场景中丰富的光场线索的挑战,
本文提出了一种聚焦感知网络探索光场数据的方法。
该方法主要包含两个部分:令牌通信模块和聚焦感知增强策略。
其中令牌通信模块通过嵌入式令牌表示汇总建立全聚焦图片和焦点堆栈的切片级特征,
并通过令牌作为信息传递的桥梁,促进网络对空间上下文建模。
聚焦感知增强策略充分考虑不同聚焦切片对于显著性的影响,
通过判断每个散焦切片的聚焦程度,来突出不同焦点切片中
显著性区域,同时抑制非显著性区域带来的负面影响。
相比现有的方法,本文方法通过附加嵌入式令牌的方式,
对光场的整体三维场景进行了切片级的探索,
并考虑了不同散焦切片对显著性预测的贡献,
能够更有效的利用光场信息。
\\
%
%
%
%
\indent
%%%%%%%%%%%%%%%%%%%%%%%%%%%%%%%%%%%%%%%%%%%%%
%
% 第二个工作点
%
%%%%%%%%%%%%%%%%%%%%%%%%%%%%%%%%%%%%%%%%%%%%%
(2)
%
%
面对如何高效的利用光场数据中全聚焦图和焦点堆栈两个模态的差异信息,
本文提出了一种视角增强网络探索光场数据的方法。
该方法主要包含两个部分:视角增强注意力模块和感知对比学习策略。
其中视角增强注意力模块通过对两个模态做交叉注意力时引入跨模态的掩码表达,
加强了注意力权重在不同聚焦区域上的显著性表达。
感知对比学习策略考虑显著性预测的前景区域内部,与背景区域内部的一致性表达。
相比现有的光场显著性检测方法,本文方法对光场数据进行跨模态的特征融合,
充分考虑了焦点堆栈和全聚焦图对最终显著性预测的贡献,
能够产生更为鲁棒的显著性物体表达。
\BiSection{创新点}{Highlights}
%
%
%创新点应该以分条列举的形式进行提出。\par
%(1) 以预报……模型,建立了….。 \par
%(2) 应用……方法,对颗粒受力情况进行了分析。\par
%(3) ……\par
%(4) ……
%
%
(1)
通过构建切片级嵌入式令牌表示,通过令牌之间的交叉注意力和移位操作,
建立了网络对于光场三维场景的感知。
\\
%
%
%
%
\indent
(2)
通过建立全聚焦特征与焦点堆栈特征的聚焦聚焦匹配,
增强了焦点堆栈中显著物体表示,并抑制非显著的背景噪声影响。
\\
%
%
%
%
\indent
(3)
提出视角增强注意力来促进跨焦点堆栈和全聚焦特征的融合,
在进行两个模态的交叉注意计算时,使用跨模态的显著性前景表达来增强权重矩阵,
降低了由于焦点堆栈中背景聚焦引起的注意力转移。
\\
%
%
%
%
\indent
(4)
引入像素对比学习策略,使得网络在分辨像素类别的同时,也能够考虑
显著性像素区域的内在联系和与非显著性区域的整体差异。
增加了网络对显著性区域的辨识能力。
\BiSection{展望}{Prospection}
%
%展望是对该研究课题存在的不足和有待改进的说明,是对未来研究的一种期待。
%
%

得益于光场信息独有的高维数据表示,
光场显著性目标检测网络在一些复杂场景上相比RGB或RGB-D的
显著性目标检测具有更大的优势。
但是,光场的多模态数据表示,需要使用多支路的神经网络进行差异化建模,
使得网络必须具有较高的数据吞吐量和足够的网络深度,
相应也带来了更高的计算负担。
现阶段的光场显著性目标检测网络大都没有实时处理的能力。
这限制了光场数据的应用。
\\
%
%
%
%
\indent
本文将在接下来的工作中,探索使用轻量化的网络实现光场显著性目标检测。
一种是探索使用剪枝或量化等策略,来降低现有网络的计算复杂度,使网络足够轻量。
或者重新设计轻量化网络,重构光场数据处理的实现结构,
目标是在较小网络参数量和较低计算量的基础上,实现性能优异的光场显著性目标检测网络。
\\
%
%
%
%
\indent
同时,现阶段的光场显著性目标检测多是在离线数据集上进行训练和测试,
本文设想构造端到端的光场显著性网络,即在轻量化网络基础上,
用一个数据流实现从光场相机的数据采集与显著性目标检测。
为光场相机在终端设备普及创造条件。




















